%!TEX root = ../../../../tugas-akhir.tex

\subsubsection{Submodul Perkalian}
\begin{table}[h]
    \small
  \caption{Fungsi dalam submodul perkalian}
  \label{tab:bn_mul}
  \begin{tabular}{R{2.8cm}L{10.5cm}}

    \toprule
    \textbf{Header Fungsi} & |int BN_mul(BIGNUM *r, const BIGNUM *a, const BIGNUM *b, BN_CTX *ctx)| \\ \midrule
    \textit{Deskripsi}     & Mengalikan $b$ pada $a$ dan menyimpan hasilya dalam $r, (r = a * b)$.\\
    \textit{Prekondisi}    & -\\
    \textit{Return Value}  & 1 jika fungsi berhasil dilakukan dan 0 jika tidak
    \\ \bottomrule
    \textbf{Header Fungsi} & |void bn_mul_normal(BN_ULONG *r, BN_ULONG *a, int na, BN_ULONG *b, int nb)| \\ \midrule
    \textit{Deskripsi}     & Perkalian $a$ dan $b$ dengan menggunakan algoritma perkalian panjang, $a$ dan $b$ adalah array yang merepresentasikan operand perkalian.  \\
    \textit{Prekondisi}    & $na = length(a)$, $nb = length(b)$, $length(r) = na + nb$ \\
    \bottomrule
    \textbf{Header Fungsi} & |void bn_mul_recursive(BN_ULONG *r, BN_ULONG *a, BN_ULONG *b, int n2 int dna, int dnb, BN_ULONG *t)| \\ \midrule
    \textit{Deskripsi}     & Perkalian $a$ dan $b$ dengan menggunakan algoritma perkalian karatsuba.\\
    \textit{Prekondisi}    & $length(r) = length(t) = 4*n2$. $n2 = 2^k, k \in \mathbb{Z} $. $dna = length(a) - n2$. $dnb = length(b) - n2$ \\
    \bottomrule
    \textbf{Header Fungsi} & |void bn_mul_part_recursive(BN_ULONG *r, BN_ULONG *a, BN_ULONG *b, int n, int tna, int tnb, BN_ULONG *t, int *used_thr)| \\ \midrule
    \textit{Deskripsi}     & Perkalian karatsuba jika n bukan merupakan bilangan pangkat dua.\\
    \textit{Prekondisi}    & $length(r) = length(t) = 8*n2$. $tna = length(a) - n$. $tnb = length(b) - n$ \\
    \bottomrule
  \end{tabular}
\end{table}

\begin{table}[h]
    \small
  \caption{Fungsi dalam submodul perkalian}
  \label{tab:bn_mul_par}
  \begin{tabular}{R{2.8cm}L{10.5cm}}

    \toprule
    \textbf{Header Fungsi} & |void *bn_mul_recursive_thread(void *ptr)| \\ \midrule
    \textit{Deskripsi}     & Fungsi untuk pemanggilan thread baru oleh pthread.\\
    \textit{Prekondisi}    & -\\
    \bottomrule

    \textbf{Header Fungsi} & |void start_mul_recursive_thread((pthread_t *thr, recursive_args *arg, BN_ULONG *r, BN_ULONG *a, BN_ULONG *b, int n2, int dna, int dnb, BN_ULONG *tmp_thr, int *used_thr))| \\ \midrule
    \textit{Deskripsi}     & Pemanggilan thread baru, menambahkan jumlah used\_thread, serta \textit{error handling}\\
    \textit{Prekondisi}    & Sama dengan fungsi |bn_mul_recursive|.\\
    \bottomrule

    \textbf{Header Fungsi} & |int get_used_thread(int* used_thr)| \\ \midrule
    \textit{Deskripsi}     & Menggunakan mutex untuk melakukan pembacaan variabel |used_thr|.\\
    \textit{Prekondisi}    & $used_thr \neq NULL$\\
    \textit{Return Value}  & Jumlah thread yang sudah dibangkitkan. \\
    \bottomrule

    \textbf{Header Fungsi} & |void *bn_mul_normal_thread(void *ptr)| \\ \midrule
    \textit{Deskripsi}     & Fungsi untuk pemanggilan thread baru oleh pthread.\\
    \textit{Prekondisi}    & $used_thr \neq NULL$\\
    \bottomrule
  \end{tabular}
\end{table}

  Submodul perkalian terdapat pada file |bn/bn_mul.c|. Fungsi |bn_mul()| merupakan fungsi yang akan dipanggil untuk melakukan operasi perkalian big integer. Terdapat dua algoritma perkalian yang diimplementasikan dalam modul ini, yaitu algoritma perkalian panjang pada fungsi |bn_mul_normal()| dan algoritma perkalian karatsuba pada |bn_mul_recursive()|. Daftar fungsi yang relevan pada submodul ini dapat dilihat pada Tabel \ref{tab:bn_mul}.

  Algoritma perkalian comba hanya digunakan untuk |BIGNUM| yang memiliki panjang 4 atau 8 word. Untuk jumlah word yang lebih tinggi, perkalian comba tidak menghasilkan hasil yang lebih baik dibandingkan dengan perkalian biasa. Implementasi algoritma perkalian comba terdapat pada fungsi |bn_mul_comba4()| dan |bn_mul_comba8()| di submodul asm.

  Basis yang dipilih dalam perkalian karatsuba adalah ketika dua operand memiliki panjang 8 word. Ketika perkalian karatsuba mencapai basis, akan dipanggil algoritma perkalian comba dalam fungsi |bn_mul_comba8()|. Implementasi algoritma paralel karatsuba dalam pseudocode \ref{alg:parallel_karatsuba} diimplementasikan pada fungsi |bn_mul_recursive()| serta |bn_mul_part_recursive()|. Perbedaan source code yang telah dimodifikasi dapat dilihat pada Lampiran \ref{sec:code_modification}.

  Algoritma perkalian panjang pada fungsi |bn_mul_normal()| akan memanggil fungsi |bn_mul_add_words()| dari submodul asm. Implementasi pseudocode \ref{alg:mul_parallel} terdapat pada fungsi ini.


\subsection{Submodul Perpangkatan Dua}
    Submodul perpangkatan dua terdapat pada file |bn/bn_sqr.c|. Fungsi |BN_sqr()| merupakan fungsi yang dipanggil untuk melakukan perpangkatan dua sebuah bilangan bulat. Fungsi ini banyak digunakan pada proses perpangkatan modular mengingat perpangkatan modular menggunakan algoritma perpangkatan \textit{sliding window} seperti yang dijelaskan pada subbab XXXX.

    Fungsi |BN_sqr()| seharusnya dapat melakukan perkalian $a*a$ lebih cepat dibandingkan dengan pemanggilan fungsi |BN_mul()| dengan parameter $a$ dan $b$ yang sama. Hal ini terjadi karena adanya simetri dalam perkalian sehingga terdapat nilai yang tidak perlu dihitung berulang-ulang. Terdapat dua algoritma yang digunakan |BN_sqr()|, yaitu algoritma perkalian panjang yang dimodifikasi serta algoritma perkalian karatsuba. Algoritma perkalian karatsuba digunakan jika panjang array merupakan bilangan pangkat dua, selain itu akan digunakan algoritma perkalian panjang yang dimodifikasi.

    Tidak terdapat paralelisasi pada submodul ini, namun penulis menemukan sebuah bug yang mengurangi kinerja fungsi |BN_sqr()|. Penggunaan algoritma perkalian panjang yang dimodifikasi memang lebih cepat dibandingkan dengan algoritma perkalian panjang yang digunakan submodul perkalian, namun algoritma ini tidak lebih cepat dibandingkan dengan penggunaan algoritma perkalian karatsuba. Paralelisasi yang telah dilakukan pada perkalian karatsuba tentu lebih mempercepat lebih jauh.

    Karena itu, dilakukan modifikasi terhadap pemilihan algoritma yang digunakan pada submodul ini. Untuk panjang array yang merupakan bilangan pangkat dua, digunakan algoritma karatsuba yang dimodifikasi, selain itu digunakan algoritma karatsuba pada submodul perkalian yang telah diparalelisasi.
