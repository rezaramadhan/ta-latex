%!TEX root = ../../../../tugas-akhir.tex

\subsubsection{Submodul Penjumlahan dan Pengurangan} \label{sec:impl_add}

  \begin{table}[!h]
     \small
  \caption{Fungsi dalam submodul penjumlahan dan pengurangan}
  \label{tab:bn_add_func}
  \begin{tabular}{R{2.8cm}L{10.5cm}}
    \toprule
    \textbf{Header Fungsi} & |int BN_add(BIGNUM *r, const BIGNUM *a, const BIGNUM *b)|    \\ \midrule
    \textit{Deskripsi}     & Menjumlahkan a dan b dan menyimpan hasilnya pada r $(a+b=r)$ \\
    \textit{Prekondisi}    & -                                                            \\
    \textit{Return Value}  & 1 jika fungsi berhasil dilakukan dan 0 jika tidak
    \\ \bottomrule
    \textbf{Header Fungsi} & |int BN_sub(BIGNUM *r, const BIGNUM *a, const BIGNUM *b)|    \\ \midrule
    \textit{Deskripsi}     & Mengurangi b dari a dan menyimpan hasilnya pada r $(a-b=r)$  \\
    \textit{Prekondisi}    & -                                                            \\
    \textit{Return Value}  & 1 jika fungsi berhasil dilakukan dan 0 jika tidak
    \\ \bottomrule
    \textbf{Header Fungsi} & |int BN_uadd(BIGNUM *r, const BIGNUM *a, const BIGNUM *b)|   \\ \midrule
    \textit{Deskripsi}     & Menjumlahkan a dan b dan menyimpan hasilnya pada r $(a+b=r)$ \\
    \textit{Prekondisi}    & $a \geq 0$, $ b \geq 0$                                      \\
    \textit{Return Value}  & 1 jika fungsi berhasil dilakukan dan 0 jika tidak
    \\ \bottomrule
    \textbf{Header Fungsi} & |int BN_usub(BIGNUM *r, const BIGNUM *a, const BIGNUM *b)|   \\ \midrule
    \textit{Deskripsi}     & Mengurangi b dari a dan menyimpan hasilnya pada r $(a-b=r)$  \\
    \textit{Prekondisi}    & $a \geq 0$, $b \geq 0$, $a \geq b$                           \\
    \textit{Return Value}  & 1 jika fungsi berhasil dilakukan dan 0 jika tidak
    \\ \bottomrule
  \end{tabular}
\end{table}

  \begin{table}[!h]
     \small
  \caption{Fungsi tambahan untuk paralelisasi penjumlahan dan pengurangan}
  \label{tab:bn_add_func_par}
  \begin{tabular}{R{2.8cm}L{10.5cm}}
    \toprule
    \textbf{Header Fungsi} & |void *bn_add_sub_words_thread(void *ptr)|    \\ \midrule
    \textit{Deskripsi}     & Fungsi untuk pemanggilan thread baru oleh pthread. \\
    \textit{Prekondisi}    & -                                                            \\ \bottomrule
    \textbf{Header Fungsi} & |void bn_resolve_carry(BN_ULONG carry, add_sub_args* arg)|    \\ \midrule
    \textit{Deskripsi}     & Menjumlahkan nilai carry yang didapat dari thread[i] ke array of |BN_ULONG| hasil penjumlahan di thread[i+1] \\
    \textit{Prekondisi}    & -                                                            \\ \bottomrule
    \textbf{Header Fungsi} & |void bn_resolve_borrow(BN_ULONG borrow, add_sub_args* arg)|    \\ \midrule
    \textit{Deskripsi}     & Menjumlahkan nilai borrow yang didapat dari thread[i] ke array of |BN_ULONG| hasil penjumlahan di thread[i+1] \\
    \textit{Prekondisi}    & -                                                            \\ \bottomrule
  \end{tabular}
\end{table}

  Submodul penjumlahan dan pengurangan terdapat pada file |bn/BN_add.c|. Fungsi |BN_add()| pada adalah melakukan pengolahan data pada a dan b seperti mengecek kenegatifan dan mengecek panjang masing-masing array. Jika a dan b memiliki tanda yang berbeda, akan dipanggil fungsi |BN_usub()| selain itu akan dipanggil fungsi |BN_uadd()|.

  |BN_uadd()| dan |BN_usub()| melakukan pengolahan data pada a dan b sehingga terdapat dua representasi array yang dapat diolah oleh |BN_add_words()| dan |BN_sub_words()|. Daftar fungsi yang terdapat pada submodul penjumlahan dan pengurangan dapat dilihat pada Tabel \ref{tab:bn_add_func}.

  |BN_add_words()| dan |BN_sub_words()| bukan merupakan bagian dari submodul ini, namun dalam submodul asm. Dua fungsi tersebut menerima masukan dua array dengan ukuran yang sama dan menjumlahkannya secara sekuensial.

  Implementasi pseudocode \ref{alg:parallel_add} terdapat pada submodul ini. Implementasi tersebut terdapat pada fungsi |BN_uadd()| dan |BN_usub()|. Dalam impementasi, dibuat tiga fungsi tambahan, yaitu |bn_add_sub_words_thread()|, |bn_resolve_carry()|, serta |bn_resolve_borrow()|. Penjelasan lebih lanjut mengenai tiga fungsi ini dapat dilihat pada tabel \ref{tab:bn_add_func_par}. Perbedaan source code asli dengan source code yang telah dimodifikasi dapat dilihat pada Lampiran \ref{sec:code_modification}.
