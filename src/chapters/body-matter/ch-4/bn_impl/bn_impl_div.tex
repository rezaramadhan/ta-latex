%!TEX root = ../../../../tugas-akhir.tex

\subsubsection{Submodul Pembagian} \label{sec:impl_div}
  Submodul pembagian terdapat pada file |bn/bn_div.c|. Fungsi |BN_div()| merupakan implementasi algoritma pembagian panjang pada OpenSSL. Sesuai dengan penjelasan pada subbab \ref{sec:div_theory}, algoritma pembagian panjang membutuhkan divisor yang telah ternormalisasi. Proses normalisasi pada implementasi ini dilakukan dengan melakukan \textit{left shift} pada divisor sehingga bit paling signifikan yang dimiliki divisor bernilai 1. Proses normalisasi divisor dilakukan oleh fungsi |bn_left_align()|, sementara itu proses \textit{left shift} pada bilangan dibagi dilakukan oleh |bn_lshift_fixed_top()| yang terdapat pada submodul |bn_shift|.
  % \todo[inline]{cek lagi istilahnya, sesuaikan sama bab 2 (divisor, number, quotient, remainder)}

  \begin{table}[h]
    \caption{Fungsi dalam submodul pembagian}
    \begin{tabular}{R{2.8cm}L{10.5cm}}
      \toprule
      \textbf{Header Fungsi} & |BN_div(BIGNUM *dv, BIGNUM *rm, const BIGNUM *num, const BIGNUM *divisor, BN_CTX *ctx)|                                                                                                       \\ \midrule
      \textit{Deskripsi}     & Membagi num dengan divisor, hasil pembagian disimpan sebagai dv dan sisa pembagian disimpan sebagai rm. Baik div maupun rm bisa menjadi NULL jika hasil atau sisa pembagian tidak dibutuhkan. \\
      \textit{Prekondisi}    & - \\
      \textit{Return Value}  & 1 jika fungsi berhasil dilakukan dan 0 jika tidak
      \\ \bottomrule
      \textbf{Header Fungsi} & |int bn_left_align(BIGNUM *num)|                                                                                                                                                              \\ \midrule
      \textit{Deskripsi}     & Normalisasi divisor BIGNUM $num$ agar $num > \beta/2$  \\
      \textit{Prekondisi}    & - \\
      \textit{Return Value}  & 1 jika fungsi berhasil dilakukan dan 0 jika tidak
      \\ \bottomrule
    \end{tabular}
  \end{table}
