\chapter{Pendahuluan}


\section{Latar Belakang}

% basically overview TLS, apa itu TLS
\textit{Transport Layer Security} (TLS) merupakan sebuah protokol yang berguna untuk menjamin keamanan dan kerahasiaan dari sebuah pesan yang dikirim melalui jaringan. Saat ini, TLS umum digunakan sebagai sebuah \textit{security layer} dari berbagai protokol yang saat ini umum digunakan di Internet seperti \textit{Hyper-Text Transfer Protocol}, (HTTP) \textit{File Transfer Protocol} (FTP), ataupun \textit{Simple Mail Transfer Protocol} (SMTP). Penggunaan TLS sebagai \textit{layer} tambahan membuat TLS dapat digunakan untuk berbagai protokol dimana keamanan informasi merupakan hal yang penting.

% Apa itu TLS Handshake, masalah TLS handshake sekarang
Sebuah koneksi TLS akan didahului dengan adanya TLS \textit{handshake} untuk menentukan \textit{symmetric key} dan \textit{cipher} yang akan digunakan oleh \textit{client} dan \textit{server} untuk berkomunikasi setelahnya. Sayangnya, waktu yang digunakan untuk TLS \textit{handshake} masih cukup signifikan; penggunaan TLS pada HTTP dapat meningkatkan waktu \textit{load} pada sebuah \textit{website} menjadi 1,2 hingga 3 kali lebih lambat dibandingkan dengan \textit{website} yang tidak menggunakan TLS \citep{cost_of_s}. Hal ini merupakan salah satu alasan kenapa saat ini TLS belum digunakan oleh seluruh layanan yang ada di internet.

% Show bagian mana dari TLS yang bikin lambat, kenalin ke big integer
Penggunaan alogritma kriptografi kunci publik pada dalam TLS Handshake membutuhkan waktu komputasi yang relatif lebih besar dibandingkan dengan proses lain. Waktu yang digunakan oleh komputasi kriptografi kunci publik sendiri adalah 13-59\% dari seluruh proses komputasi pada TLS \citep{perf_tls}. Komputasi kriptografi kunci publik perlu menggunakan bilangan bulat yang besar untuk memastikan kunci yang digunakan dalam kriptografi aman. Komputasi bilangan bulat besar (\textit{big integer}) tersebutlah yang menyebabkan komputasi kriptografi kunci publik  memerlukan waktu yang lama.

% Metode yang disarankan untuk meningkatkan TLS handshake
% Ini perlu nyantumin paper engga?
Terdapat beberapa metode yang telah disarankan untuk meningkatkan kinerja TLS \textit{handshake} seperti melakukan optimisasi pada algoritma Rivest–Shamir–Adleman (RSA) dengan pemrograman paralel, melakukan \textit{batching} pada proses RSA di server, ataupun melakukan \textit{rebalancing} sehingga komputasi pada client lebih sulit daripada server. Solusi lain yang dapat digunakan adalah memparalelkan komputasi \textit{big integer} sehingga komputasi tersebut dapat dilakukan dalam waktu lebih pendek.

% Solusi big integer, berbagai algoritma yang ada dan kemungkinannya untuk diparalelkan
% Montgomery italic ga? nama orang
Komputasi big integer bukanlah hal yang baru dalam dunia ilmu komputer. \citet{modern_comp_math} memberikan tujuh algoritma yang dapat digunakan dalam perkalian, delapan algoritma pembagian, serta berbagai alternatif algoritma  yang dapat digunakan dalam setiap operasi dalam komputasi big integer. Terdapat beberapa juga beberapa algoritma yang dapat dijalankan secara paralel, seperti metode representasi \textit{Residue Number Systems} dan algoritma Montgomery pada komputasi perkalian modular.

Terdapat dua pilihan yang dapat digunakan untuk menjalakan komputasi \textit{big integer} secara paralel, yaitu menjalakan komputasi pada CPU atau GPU. Penggunaan GPU dalam memparalelkan komputasi kriptografi kunci publik dapat menghasilkan jumlah \textit{transaction per second} hingga 45 kali lebih besar dibandingkan dengan penggunaannya pada CPU \citep{gpu_vs_cpu}. Namun, GPU tidak umum digunakan pada data center sehingga penggunaan GPU dalam komputasi kriptografi kunci publik pada kehidupan sehari-hari tidak realistis. Dengan umumnya \textit{multicore} CPU yang digunakan server, penggunaan CPU dalam paralelisasi lebih realistis.


\section{Rumusan Masalah}

Sesuai dengan pembahasan yang ada pada latar belakang, diketahui bahwa penggunaan pemrograman paralel dalam komputasi \textit{big integer} merupakan salah satu metode yang dapat digunakan untuk meningkatkan kinerja TLS \textit{Handshake}. Maka dari itu, tugas akhir ini akan fokus pada pengembangan algoritma paralel yang digunakan dalam komputasi \textit{big integer} dan implementasinya pada TLS \textit{handshake}. Dalam pengembangannya, terdapat berapa masalah yang akan menjadi perhatian utama dalam tugas akhir ini, yaitu:

\begin{enumerate}
  \item Algoritma paralel seperti apa yang akan diimplementasikan dalam komputasi \textit{big integer}?
  \item Bagaimana cara mengintegrasikan algoritma paralel tersebut ke dalam library \textit{big integer}?
  \item Apakah penggunaan algoritma paralel dalam komputasi \textit{big integer} dapat meningkatkan kinerja dari TLS \textit{handshake}?
\end{enumerate}

\section{Tujuan}

Dalam penyelesaian tugas akhir ini terdapat beberapa tujuan yang akan dicapai, yaitu:
\begin{enumerate}
  \item Menemukan algoritma paralel yang dapat digunakan dalam komputasi \textit{big integer}.
  \item Mengintegrasikan algoritma tersebut dalam sebuah implementasi library \textit{big integer}.
  \item Mengukur peningkatan kinerja dari TLS \textit{handshake} setelah mengimplementasikan algoritma paralel RSA.
\end{enumerate}

\section{Batasan Masalah}

Tugas akhir ini akan membahas mengenai optimisasi algoritma RSA pada sebuah implementasi TLS \textit{handshake}. Adapun batasan masalah yang terdapat pada tugas akhir ini adalah:

\begin{enumerate}
  \item Paralelisasi library big integer tidak memperhatikan masalah keamanan pada sistem yang menggunakannya.
  \item Implementasi TLS memanfaatkan perangkat lunak opensource OpenSSL versi 1.1.1.
  \item Ciphersuite yang digunakan dalam penelitian ini adalah TLS\_DH\_RSA\_WITH\_AES\_256\_CBC\_SHA256.
  \item CPU yang digunakan pada implementasi ini adalah Intel 64-bit yang minimal memiliki enam belas \textit{core}.
\end{enumerate}

\section{Metodologi}

Pengerjaan tugas akhir ini akan melalui beberapa tahap, yaitu:
\begin{enumerate}
  \item Analisis Masalah

  Tugas akhir akan dimulai dengan dilakukannya analisis mengenai masalah terhadap implementasi TLS \textit{handshake} yang ada pada saat ini. Pada tahap ini akan dipelajari juga mengenai solusi-solusi untuk meningkatkan kinerja dari TLS \textit{handshake} yang sudah tersedia pada saat ini. Dari solusi yang sudah ada tersebut, penulis akan mencari penyebab solusi itu belum diimplementasikan pada TLS \textit{handshake} dan mencari solusi baru yang mungkin untuk diimplementasikan.

  \item Perancangan Solusi dan Pengujian

  Perancangan solusi dibuat sebagai kerangka dasar atas solusi dari masalah yang telah ditemukan pada langkah sebelumnya. Pada tahap ini akan dirancang arsitektur solusi beserta cara untuk menerapkannya pada TLS handshake yang sudah ada. Selain itu, akan dibuat juga rencana pengujian terhadap implementasi solusi. Rencana pengujian akan dibuat sehingga percobaan dilakukan dalam batas masalah yang ada dan dapat dilakukan sesuai metode sains.

  \item Implementasi Solusi

  Pada tahap ini, penulis akan mengimplementasikan rancangan solusi yang sudah dibuat dalam sebuah bentuk perangkat lunak yang dapat digunakan dalam dunia sehari-hari. Hasil implementasi yang dibuat tentu ditujukan untuk menyelesaikan masalah yang ada.

  \item Pengujian dan Evaluasi

  Pengujian akan dilakukan berdasarkan rancangan pengujian telah ada. Pengujian dilakukan untuk mengetahui apakah solusi yang diberikan merupakan solusi terbaik dari masalah yang ada. Hasil dari pengujian akan digunakan sebagai data evaluasi dan penarikan kesimpulan.

\end{enumerate}


\section{Sistematika Laporan}
\todo[inline]{Diisi nanti kalo laporan udah beres}

% \blindtext
% Penulisan tugas akhir ini akan terdiri dari lima bab yaitu: BAB I Pendahuluan, BAB II Studi Literatur, BAB III Analisis Solusi, BAB IV Rancangan, Implementasi dan Pengujian, BAB V Penutup.
%
% Bab satu memaparkan mengenai latar belakang pembuatan tugas akhir ini, masalah utama yang dibahas, tujuan pembuatan tugas akhir, serta metodologi yang akan dilakukan selama pembuatan tugas akhir. Bab ini dibuat sebagai pendahuluan dan akan mencakup ringkasan dari sebagian besar hal yang dibahas pada tugas akhir.
%
% Bab dua akan membahas mengenai teori dasar yang sudah ada mengenai protokol TLS, algoritma RSA, serta pengembangan dari algoritma RSA melalui pemrograman paralel yang sudah ada. Teori yang diambil akan bersumber dari literatur terpercaya serta dokumentasi resmi dari protokol TLS.
%
% Bab tiga menggambarkan solusi yang digunakan untuk menyelesaikan masalah yang ada, kenapa digunakan solusi tersebut, serta hipotesa peningkatan kinerja dari TLS handshake setelah solusi yang dibuat diimplementasikan.
%
% Bab empat memperlihatkan rancangan implementasi dari solusi, proses implementasi solusi tersebut, serta hasil pengujian dari implementasi tersebut pada kinerja TLS handshake.
%
% Bab lima mendeskripsikan hasil kesimpulan dari solusi yang dibuat untuk menyelesaikan masalah serta saran yang dapat dilakukan untuk pengembangan dan perbaikan agar implementasi ini dapat menghasilkan hasil yang lebih baik.
