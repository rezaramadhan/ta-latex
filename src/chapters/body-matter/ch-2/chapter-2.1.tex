\section{Big Integer}
% Arithmetic di komputer
Komputasi matematis di dalam sebuah komputer dilakukan oleh Arithmetic Logic Unit (ALU) yang terdapat di dalam CPU. Karena ALU adalah komponen yang sangat sederhana, ALU memiliki banyak batasan, salah satunya adalah ALU hanya dapat beroperasi pada bilangan bulat dalam \textit{range} tertentu \citep{comp_org_arch}. Pada umumnya, sebuah CPU memiliki 32-bit atau 64-bit ALU, nilai maksimum bilangan bulat yang dapat diproses oleh ALU tersebut hanya sebesar $2^{64}$. Kemampuan ALU untuk menangani bilangan pada \textit{range} tertentu berdasarkan jumlah bit yang dimilikinya biasa dikenal sebagai \textit{fixed-precision integer arithmetic}.

% Big number, Apa itu, kenapa dibutuhkan, dikenal juga sebagai Arbitrary/multi precision
Untuk menangani operasi matematis yang menggunakan bilangan yang lebih besar dari \textit{range} yang dimiliki ALU, diperlukan sebuah struktur data yang dapat menangani bilangan bulat tersebut. Kemampuan komputer untuk menghitung bilangan yang tidak memiliki batas sering dikenal sebagai \textit{arbitrary-precision integer arithmetic}. Sementara itu, bilangan yang digunakan dalam perhitungan tersebut sering disebut \textit{big number} atau \textit{big integer} jika bilangan yang digunakan adalah bilangan bulat.

% dipake dimana aja
\textit{Big integer} sering digunakan pada perhitungan kriptografi, mengingat bahwa perhitungan kriptografi membutuhkan bilangan yang besar agar kunci yang digunakan aman. Untuk implementasi kriptografi yang aman, disarankan untuk menggunakan kunci sebesar 256bit untuk AES dan kunci sebesar 1024 bit untuk RSA \citep{key_suggestion}, lebih besar dari jumlah bit yang dimiliki oleh ALU. Selain perhitungan kriptografi, big number juga umum digunakan untuk menghitung nilai konstanta matematis seperti $\pi$ \citep{bn_pi}, .
% \blindtext
\subsection{Struktur Big Integer}
% Representasi logical
\citet{modern_comp_math} menyatakan bahwa sebuah bilangan bulat dapat direpresentasikan sebagai penjumlahan dari komponen-komponennya. Jika kita memilih sebuah bilangan bulat positif $\beta$ sebagai basis dengan $\beta > 1 $ semua bilangan bulat positif $A$ berbasis $\beta$ yang memiliki panjang $n$ dapat dituliskan sebagai:
\begin{equation}
  A = \alpha_{n-1}\beta^{n-1}+...+\alpha_{1}\beta+\alpha_{0}
\end{equation}
dengan $0 \leq \alpha \leq \beta -1$.

Representasi integer positif di komputer 64 bit menggunakan $\beta = 2$ dan $n = 32$ sehingga nilai maksimum yang dapat direpresentasikan adalah $2^{64}$. Untuk representasi big integer nilai $n$ tidak memiliki batas, sementara nilai $\beta$ yang digunakan sesuai dengan jumlah maksimum bilangan dapat diproses pada komputer tersebut. Pada representasi big integer di komputer 64 bit, digunakan $\beta = 2^{64}$.

% Physical yang umum, array
% masuk bab 3?
Representasi ideal big integer pada komputer adalah menggunakan list of integer atau array of integer. Penggunaan list menyebabkan big integer tidak memiliki nilai maksimum, sementara itu penggunaan array membuat akses nilai yang tersimpan lebih cepat. Dengan demikian, penggunaan array of integer lebih umum digunakan untuk merepresentasikan big integer. Ilustrasi array of integer yang merepresentasikan big integer dapat dilihat pada Gambar XXX. 
\colorbox{BurntOrange}{Tambahin gambar disini, kotak2 array biasa aja}
% \blindtext
\subsection{Operasi}
% \blindtext
\subsubsection{Penjumlahan dan Penguragan}
% \blindtext
\subsubsection{Perkalian}
% \blindtext
\subsubsection{Pembagian}
% \blindtext
\subsubsection{Modulo}
% \blindtext
\subsubsection{Perpangkatan}
% \blindtext
\subsubsection{Perkalian Modulo}
% \blindtext
\subsubsection{GCD}
% \blindtext
