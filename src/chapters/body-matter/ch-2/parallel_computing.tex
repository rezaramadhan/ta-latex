\section{Komputasi Paralel}
  % Apa itu
  Komputasi paralel adalah sebuah mode komputasi dimana banyak komputasi dijalankan pada waktu yang sama \citep{highly_parallel_computing}. Sebuah proses pada komputasi paralel biasanya dibagi menjadi beberapa subproses kecil yang akan dijalankan secara independen. Pembagian sebuah proses menjadi subproses merupakan tantangan utama dalam pemrograman paralel.

  % Kenapa butuh -> perkembangan processor ke arah sana
  Pada awalnya, komputasi paralel hanya digunakan pada superkomputer mengingat keperluan \textit{hardware} yang tinggi yang diperlukan untuk melakukan komputasi paralel. Namun, setelah IBM memperkenalkan IBM POWER4 sebagai sebuah processor multicore pertama pada tahun 2001, komputasi paralel dapat dilakukan oleh komputer komersial umum \citep{power4}. Intel Platinum D yang diluncurkan pada tahun 2005 merupakan awal dari jajaran multicore processor yang sekarang sudah umum digunakan pada laptop, komputer dekstop, ataupun smartphone yang kita gunakan.
  % \todo{tambahin kutipan}

  Perkembangan processor pada dua dekade ke belakang berfokus ke pembuatan jumlah core yang banyak dibandingkan dengan perkembagan kinerja dari single core. Pada tahun 1986 hingga 2002, kinerja single core processor meningkat 50\% per tahun \citep{comp_arch_patterson}. Sementara itu, peningkatan kinerja single core dari tahun 2002 hanya meningkat sebesar 20\% \citep{intro_parallel} dan akan semakin berkurang pada setiap tahunnya.
  % \todo{Tambahin data dari performance increase / intel CPU}

  % Kelebihan
  Komputasi paralel umum digunakan untuk memroses banyak data yang tidak berkaitan satu sama lain. Karena itulah komputasi paralel banyak digunakan pada bidang kecerdasan buatan untuk melakukan \textit{training} pada model yang dibutuhkan. Selain pemrosesan data, komputasi paralel unggul dalam melakukan komputasi aritmatika sederhana seperti perhitungan vektor dalam \textit{3D rendering}. Komputasi paralel juga banyak digunakan dalam komputasi pelipatan protein, permodelan iklim, pembuatan obat, serta penelitian energi \citep{intro_parallel}.
  % \todo{tambahin kutipan}

  % Apa yang perlu diperhatikan dalam paralel programming
  Pada komputasi paralel, terdapat beberapa hal yang perlu diperhatikan yang biasanya tidak menjadi masalah pada komputasi biasa. \textit{Memory management} perlu lebih diperhatikan dibandingkan dengan pemrograman biasa. Penggunaan \textit{memory management} yang tidak baik dapat menyebabkan program tidak berjalan dengan optimal karena program tidak memperhatikan penggunaan cache dengan baik, bahkan program bisa tidak berjalan sama sekali jika terjadi \textit{deadlock}. Selain itu, \textit{debugging} pada program lebih sulit dilakukan karena \textit{programmer} perlu memperhatikan beberapa \textit{routine} yang berjalan secara paralel. Struktur kode yang tidak rapi akan membuat debugging menjadi lebih susah.

  % Term
  Terdapat beberapa terminologi yang umum digunakan dalam komputasi paralel. Beberapa terminologi yang digunakan dalam penelitian ini diantaranya.
  \begin{itemize}
    \item \textit{Task}, yaitu bagian diskrit dalam sebuah komputasi. Task biasanya terdiri dari kumpulan instruksi yang dieksekusi oleh sebuah processor.  \citep{intro_parallel_llnl}
    \item \textit{Job}, yaitu sebuah process atau thread yang melakukan sebuah task secara independen. Dalam komputasi paralel, kumpulan job akan berjalan pada beberapa process namun mengerjakan komputasi yang sama. \citep{glossary_udel}
    \item Sinkronisasi, yaitu koordinasi dari beberapa job yang berjalan secara paralel \citep{intro_parallel_llnl}. Koordinasi dilakukan melalui \textit{message passing, lock, mutex,} ataupun \textit{semaphore}.
    \item \textit{Speedup}, yaitu berapa kali lebih cepatnya sebuah program paralel dibandingkan dengan versi sekuensial dari program tersebut. \citep{glossary_ufl}
    \item Efisiensi, yaitu perbandingan \textit{speedup} dengan jumlah job yang digunakan dalam sebuah program.

  \end{itemize}


  % Pitfalls
  \citep{structured_parallel_programming} menyatakan bahwa terdapat beberapa hal yang harus dihindari dalam pembuatan program secara paralel. Hal tersebut dapat terjadi karena sinkronisasi antar \textit{job} tidak dilakukan dengan baik. Terlalu banyak sinkronisasi dapat menyebabkan \textit{scaling} program sulit, sementara itu sinkronisasi yang kurang dapat menyebabkan hasil yang dihasilkan program tidak konsisten. Berikut adalah hal yang perlu diperhatikan.
  \begin{itemize}
    \item \textit{Race condition}, yaitu kondisi dimana hasil yang dihasilkan dari komputasi paralel tidak konsisten karena kurangnya sinkronisasi antar \textit{job}. Sebagai contoh, job A menulis nilai baru pada memori M, sementara itu job B membaca memori M sebelum job A menulis nilai baru tersebut ketika job B membutuhkan nilai memori M yang terbaru.
    \item \textit{Deadlock} terjadi ketika dua \textit{job} menunggu satu sama lain untuk menggunakan sebuah resource yang sama.
    \item \textit{Scaling} program lebih susah, hal ini terjadi karena sinkronisasi yang terlalu tinggi. Karena sinkronisasi antar job sangat tinggi, program tidak berbeda jauh dengan program yang dijalankan secara sekuensial. Jika program menggunakan lock dan mutex, bisa jadi waktu operasi lock tersebut justru memakan waktu yang lebih lama dibandingkan dengan job yang perlu dijalankan.
    \item \textit{Load imbalance} antar job.
    \item \textit{Overhead} pembuatan job. Jika jumlah job yang dibuat program terlalu banyak, waktu yang diperlukan dalam pembuatan dan penghancuran job menjadi semakin signifikan. Perlu dicari jumlah job yang cukup sehingga overhead tidak terlalu besar namun program tetap dijalankan secara paralel dengan efektif.
  \end{itemize}
