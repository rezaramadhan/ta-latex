%!TEX root = ../../tugas-akhir.tex
\clearpage
\paragraph{Kasus Uji Fungsional} \label{sec:functional_testcase}

Berikut adalah contoh kasus uji yang digunakan dalam pengujian fungsional modul bn dalam OpenSSL. Test uji yang sebenarnya digunakan tidak dimasukkan pada bagian ini karena jumlahnya terlalu banyak.

\subparagraph{Penjumlahan dan Pengurangan}
\VerbatimInput[label=\fbox{bn\_sum.txt}]{resources/text/bn_testcase/bn_sum.txt}

\subparagraph{Perkalian}
\VerbatimInput[label=\fbox{bn\_mul.txt}]{resources/text/bn_testcase/bn_mul.txt}

\subparagraph{Pembagian}
\VerbatimInput[label=\fbox{bn\_div.txt}]{resources/text/bn_testcase/bn_div.txt}

\subparagraph{Perpangkatan}
\VerbatimInput[label=\fbox{bn\_exp.txt}]{resources/text/bn_testcase/bn_exp.txt}
%
\subparagraph{Perkalian Modular}
\VerbatimInput[label=\fbox{bn\_modmul.txt}]{resources/text/bn_testcase/bn_modmul.txt}
%
\subparagraph{Perpangkatan Modular}
\VerbatimInput[label=\fbox{bn\_modexp.txt}]{resources/text/bn_testcase/bn_modexp.txt}
