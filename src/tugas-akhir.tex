%--------------------------------------------------------------------%
%
% Berkas utama templat LaTeX.
%
% author Petra Barus, Peb Ruswono Aryan
%
%--------------------------------------------------------------------%
%
% Berkas ini berisi struktur utama dokumen LaTeX yang akan dibuat.
%
%--------------------------------------------------------------------%

\documentclass[11pt, a4paper, onecolumn, oneside, final]{book}

%-------------------------------------------------------------------%
%
% Konfigurasi dokumen LaTeX untuk laporan tesis IF ITB
%
% @author Petra Novandi
%
%-------------------------------------------------------------------%
%
% Berkas asli berasal dari Steven Lolong
%
%-------------------------------------------------------------------%




%===============================================================================
% Required Package
%===============================================================================
\usepackage{etoolbox}                % Latex Command modification
\usepackage[dvipsnames]{xcolor}      % Colors
\usepackage{graphicx}                % Fancy Graphics
\usepackage{blindtext}               % Lorem ipsum filling
\usepackage{chngcntr}                % Set counter
\usepackage{parskip}                 % Space between paragraph
\usepackage{ragged2e}                % Justify paragraph
\usepackage{caption}                 % Caption on figures, tables, etc
\usepackage{enumitem}                % Cuztomize enumerate & itemize
\usepackage{textcase}                % MakeTextUppercase & lower
\usepackage{setspace}                % Set text spacing
\usepackage{afterpage}               % Figures on the next page
\usepackage{amsmath,amsfonts}        % Latex math typesetting
\usepackage{todonotes}               % Todo in document
\usepackage{relsize}                 % Change text size
\usepackage{pgfplots}                % Plot & graph
\usepackage{pdfpages}
% \usepackage{titling}               % Make title



%===============================================================================
% Document Configurations
%===============================================================================
\renewcommand{\baselinestretch}{1.5}                          % Line Spacing
\special{papersize=210mm,297mm}                               % Paper Size
\usepackage[top=3cm,bottom=4cm,left=4cm,right=3cm]{geometry}  % Margin
% \usepackage[bahasa]{babel}                                    % Bahasa
\usepackage[utf8]{inputenc}                                   % encoding
\usepackage{times}                                            % Font

\makeatletter
\renewcommand\Large{\@setfontsize\large{14pt}{18}}
\makeatother


%===============================================================================
% titlesec (judul bab & subbab) Configuration
%===============================================================================
% Required Package
\usepackage{titlesec}

% Chapter style
% \chapterfont{\centering \large}
\titleformat{\chapter}[display]
  {\large\centering\bfseries}
  {\chaptertitlename\ \thechapter}{0pt}
    {\large\bfseries\uppercase}
\titlespacing*{\chapter}{0pt}{-25pt}{40pt}

% section
\titleformat*{\section}{\bfseries\normalsize}
\titleformat*{\subsection}{\bfseries\normalsize}
\titleformat{\subparagraph}
    {\normalfont\normalsize\bfseries}{\thesubparagraph}{1em}{}
\titlespacing*{\section} {0pt}{2.5ex plus 1ex minus .2ex}{1.3ex plus .2ex}
\titlespacing*{\subsection} {0pt}{2.25ex plus 1ex minus .2ex}{0.5ex plus .2ex}
\titlespacing*{\subparagraph}{\parindent}{3.25ex plus 1ex minus .2ex}{.75ex plus .1ex}

% paragraph, used in appendix
\renewcommand{\theparagraph}{\Alph{paragraph}. }        % set counter
\titleformat{\paragraph}[block]                         % paragraph style
  {\normalsize\bfseries}
  {Lampiran \theparagraph}{0pt}
    {\normalsize\bfseries}
\titlespacing*{\paragraph}
{0pt}{3.25ex plus 1ex minus .2ex}{1.5ex plus .2ex}



%===============================================================================
% titletoc Configuration
%===============================================================================
\usepackage{titletoc}


\renewcommand{\contentsname}{DAFTAR ISI}
\renewcommand{\listfigurename}{DAFTAR GAMBAR}
\renewcommand{\figurename}{Gambar}
\renewcommand{\tablename}{Tabel}

\titlecontents{chapter}[0pt]
              {\bfseries}
              {BAB \hspace*{1.8em} \contentslabel{2em}}{}
              {\titlerule*[0.35pc]{.}\contentspage}

\titlecontents{section}[3em] % ie, 1.5em (chapter) + 2.3em
              {}
              {\contentslabel{2.3em}}{}
              {\titlerule*[0.35pc]{.}\contentspage}
\titlecontents{subsection}[6em] % ie, 1.5em (chapter) + 2.3em
              {}
              {\contentslabel{2.9em}}{}
              {\titlerule*[0.35pc]{.}\contentspage}
\titlecontents{subsubsection}[9em] % ie, 1.5em (chapter) + 2.3em
              {}
              {\contentslabel{3.6em}}{}
              {\titlerule*[0.35pc]{.}\contentspage}

% Appendices
\titlecontents{paragraph}[0em] % ie, 1.5em (chapter) + 2.3em
              {}
              {Lampiran \hspace*{1.6em} \contentslabel{1.6em}}{}
              {\titlerule*[0.35pc]{.}\contentspage}

% LoF & LoT
\contentsuse{figure}{lof}
\titlecontents{figure}[0em] % ie, 1.5em (chapter) + 2.3em
              {}
              {Gambar \hspace*{2.1em} \contentslabel{2.3em}}{}
              {\titlerule*[0.35pc]{.}\contentspage}
\contentsuse{table}{lot}
\titlecontents{table}[0em] % ie, 1.5em (chapter) + 2.3em
              {}
              {Tabel \hspace*{2.1em} \contentslabel{2.3em}}{}
              {\titlerule*[0.35pc]{.}\contentspage}


%===============================================================================
% Table of Content & sections Configuration
%===============================================================================
\setcounter{secnumdepth}{4}   % Set maximum depth of numbering on sections
\setcounter{tocdepth}{3}      % Set maximum depth of sections on ToC

% Counter untuk figure dan table.
\counterwithin{figure}{chapter}
\counterwithin{table}{chapter}


%===============================================================================
% pgfplots
%===============================================================================
\pgfplotsset{
    discard if not/.style 2 args={
        x filter/.code={
            \edef\tempa{\thisrow{#1}}
            \edef\tempb{#2}
            \ifx\tempa\tempb
            \else
                \def\pgfmathresult{inf}
            \fi
        }
    },
    every axis legend/.append style={
        at={(1.02,1)},
        anchor=north west,
    },
}
\usepgfplotslibrary{units}
\renewcommand{\axisdefaultwidth}{10cm}
\renewcommand{\axisdefaultheight}{8.5cm}




%===============================================================================
% Disable Hyphenation
%===============================================================================
\hyphenpenalty=10000
\tolerance=1
\lefthyphenmin=99
\righthyphenmin=99
\sloppy


%===============================================================================
% Bibliography Configuration
%===============================================================================
\usepackage[natbibapa]{apacite}       % citation: natbib dengan format apacite
\bibliographystyle{apacite}

% Ubah 'and' -> 'dan', 'et. al' -> 'dkk'
\renewcommand{\BBAA}{dan}  % between authors in parenthetical cites and ref. list
\renewcommand{\BBAB}{dan}  % between authors in in-text citation
\renewcommand{\BAnd}{dan}  % for ``Ed. \& Trans.'' in ref. list
\renewcommand{\BOthers}{dkk}

% Ubah References -> DAFTAR PUSTAKA
\renewcommand{\bibname}{DAFTAR PUSTAKA}

%===============================================================================
% Table Configuration
%===============================================================================
% Required package
\usepackage{booktabs}
\usepackage{longtable}
\usepackage{array}
\usepackage{multirow}
\usepackage{tabularx}

% New column type -> fixed column width with certain column alignment
\newcolumntype{L}[1]{>{\raggedright\let\newline\\\arraybackslash\hspace{0pt}}m{#1}}
\newcolumntype{C}[1]{>{\centering\let\newline\\\arraybackslash\hspace{0pt}}m{#1}}
\newcolumntype{R}[1]{>{\raggedleft\let\newline\\\arraybackslash\hspace{0pt}}m{#1}}

\renewcommand{\arraystretch}{1.2}         % space between table border & content
\renewcommand{\lightrulewidth}{0.02em}    % width of light (middle) table border
\renewcommand{\heavyrulewidth}{0.1em}     % width of heavy (top & bottom) table border


%===============================================================================
% listing (code) Configuration
%===============================================================================
% Required package
\usepackage{listings}

% Create config for line number format
\makeatletter
\def\lst@numbersymbol{}
\lst@Key{numbersymbol}{}{\def\lst@numbersymbol{#1}}
\lst@Key{numbers}{none}{%
    \let\lst@PlaceNumber\@empty
    \lstKV@SwitchCases{#1}%
    {none&\\%
     left&\def\lst@PlaceNumber{\llap{\normalfont
                \lst@numberstyle{\thelstnumber\lst@numbersymbol}\kern\lst@numbersep}}\\%
     right&\def\lst@PlaceNumber{\rlap{\normalfont
                \kern\linewidth \kern\lst@numbersep
                \lst@numberstyle{\lst@numbersymbol\thelstnumber}}}%
    }{\PackageError{Listings}{Numbers #1 unknown}\@ehc}}
\makeatother

% Change 'Listing' -> 'Source Code'
\renewcommand{\lstlistingname}{\textit{Source Code}}
% 'List of Listing' -> 'Daftar Source Code'
\renewcommand{\lstlistlistingname}{Daftar of \lstlistingname}

% listing Style
\lstset{
    float=tp,
    floatplacement=tbp,
    abovecaptionskip=0pt,
    belowcaptionskip=10pt,
    breaklines=true,
    breakatwhitespace=false,
    frame=single,
    xleftmargin=2em,
    framexleftmargin=1.5em,
    numbersep=1.5em,
    numbers=left,
    numbersymbol=:,
    numberstyle=\tiny\color{gray!75!black},
    stepnumber=1,
    language=C,
    basicstyle=\linespread{1.15}\footnotesize\ttfamily\color{black},
    keywordstyle=\bfseries\color{green!40!black},
    commentstyle=\itshape\color{purple!40!black},
    stringstyle=\color{orange},
}
\lstMakeShortInline[columns=fixed]|                 % Inline listing



%===============================================================================
% fancyvrb (code) Configuration
%===============================================================================
% Required Package
\usepackage{fancyvrb}

% fancyvrb style
\RecustomVerbatimCommand{\VerbatimInput}{VerbatimInput}{
  fontsize=\relsize{-2},
  baselinestretch=0.75,
  frame=lines,                            % top and bottom rule only
  framesep=1.5em,                         % separation between frame and text
  labelposition=topline,
}

%===============================================================================
% hyperref must be (one of) the last package to be loaded
%===============================================================================
\usepackage{hyperref}                % Link di daftar isi.
\usepackage{hypcap}                  % Link captions



%===============================================================================
% algorithmicx (pseudocode) Configuration
%===============================================================================

% Required package
\usepackage[chapter]{algorithm}
\usepackage{algorithmicx}
\usepackage[noend]{algpseudocode}

% Change 'Algorithm' -> Pseudocode
\makeatletter
  \renewcommand{\ALG@name}{Pseudocode}
\makeatother

\newcommand*\Let[2]{\State #1 $\gets$ #2}
\MakeRobust{\Call}
\algrenewcommand\algorithmicrequire{\textbf{Prekondisi:}}
\algrenewcommand\algorithmicensure{\textbf{Postcondition:}}



\begin{document}

    %-------------------------------------------------------------------------%
    % Set Variable
    %-------------------------------------------------------------------------%
    \newcommand{\thetitle}
      {Paralelisasi Library big integer pada Transport Layer Security (TLS) Handshake}
    \newcommand{\thedatemy}
      {Oktober 2019}
    \newcommand{\thedatedmy}
      {8 Oktober 2019}
    \newcommand{\theauthor}
      {Muhammad Reza Ramadhan}
    \newcommand{\thestudentnumber}
      {13514107}


    %-------------------------------------------------------------------------%
    % Front Matter
    %-------------------------------------------------------------------------%
    \frontmatter
    %!TEX root = ../../tugas-akhir.tex
\clearpage
\pagestyle{empty}

\begin{center}
\smallskip

    \large \bfseries \MakeUppercase{\thetitle}
    \vfill

    \large Laporan Tugas Akhir
    \vfill

    \normalsize Disusun sebagai syarat kelulusan tingkat sarjana
    \vfill

    \normalsize Oleh

    \large \theauthor

    \large NIM: \thestudentnumber

    \vfill
    \begin{figure}[h]
        \centering
      	\includegraphics[width=0.15\textwidth,natwidth=700,natheight=1021]{resources/cover-ganesha.jpg}
    \end{figure}
    \vfill

    \large
    \uppercase{
        Program Studi Teknik Informatika \\
        Sekolah Teknik Elektro dan Informatika \\
        Institut Teknologi Bandung \\
        \thedatemy
    }

\end{center}

\clearpage

    %!TEX root = ../../tugas-akhir.tex
\clearpage
\pagestyle{empty}

\begin{center}
\smallskip

    \large \bfseries \MakeUppercase{\thetitle}
    \vfill

    \large Laporan Tugas Akhir
    \vfill

    \normalsize Oleh

    \large \theauthor

    \large NIM: \thestudentnumber

    \large Program Studi Teknik Informatika

    \normalsize \normalfont Sekolah Teknik Elektro dan Informatika \\
    Institut Teknologi Bandung \\

    \vfill
    \normalsize \normalfont
    Telah disetujui dan disahkan sebagai Laporan Tugas Akhir
    \\[0.1em]
    di Bandung, pada tanggal \thedatedmy.

    \vfill
    \setlength{\tabcolsep}{12pt}
    \begin{tabular}{c@{\hskip 0.5in}c}
        Pembimbing I, & Pembimbing II \\
        & \\
        & \\
        & \\
        & \\
        \underline{Dr. Ir. Rinaldi Munir, MT.} & \underline{Dr. Judhi Santoso M.Sc.} \\
        NIP 196512101994021001 & NIP 196302041989031002 \\
    \end{tabular}

\end{center}
\clearpage

    \input{chapters/front-matter/statement}

    \pagestyle{plain}

    %!TEX root = ../../tugas-akhir.tex
\clearpage
\chapter*{Abstrak}
\addcontentsline{toc}{chapter}{ABSTRAK}

\begin{center}
  \large \bfseries \MakeUppercase{\thetitle}

  \normalsize \normalfont Oleh

  \theauthor

  NIM : \thestudentnumber
\end{center}


%taruh abstrak bahasa indonesia di sini
\begin{singlespacing}
% \setstretch{1.0}

Transport Layer Security (TLS) merupakan sebuah protokol yang umum digunakan sebagai layer security oleh sebuah protokol lain seperti HTTP, SMTP, dan DNS. Kekurangan utama dari TLS adalah proses TLS Handshake yang memakan waktu relatif lebih lama. TLS Handshake menggunakan algoritma pertukaran kunci serta proses autentikasi melalui kriptografi kunci publik yang membutuhkan komputasi yang tinggi.

Algoritma pertukaran kunci serta kriptografi kunci publik melakukan operasi perpangkatan modulo pada bilangan bulat yang besar. Operasi aritmatika pada bilangan bulat yang besar tersebutlah yang membutuhkan proses komputasi yang tinggi. Terdapat banyak algoritma yang dapat digunakan pada operasi aritmatika tersebut, salah satu variasinya dalah menggunakan varian algoritma yang dapat dijalankan secara paralel.

Varian algoritma paralel yang dapat digunakan diantaranya adalah varian dari algoritma penjumlahan, pengurangan, perkalian panjang, perkalian karatsuba, serta perkalian modular Montgomery. Seluruh algoritma paralel tersebut diimplementasikan dalam OpenSSL v1.1.1e. Pengujian terhadap implementasi tersebut dilakukan pada setiap algoritma yang digunakan, proses pertukaran kunci Diffie-Hellman, proses pembangkitan kunci RSA, serta proses enkripsi dan dekripsi RSA.

\todo[inline]{hasil \& kesimpulan}

\end{singlespacing}


Kata Kunci: \textit{Big Integer, Transport Layer Security}, Algoritma Karatsuba

    % \input{chapters/front-matter/abstract-en}
    %!TEX root = ../../tugas-akhir.tex
\chapter*{Kata Pengantar}
\addcontentsline{toc}{chapter}{KATA PENGANTAR}

\Blindtext

\begin{flushright}
Bandung, \thedatedmy
\\[2cm]

\theauthor
\end{flushright}


    \titleformat*{\section}{\centering\bfseries\Large\MakeUpperCase}

    %-------------------------------------------------------------------------%
    % Daftar Isi, Tabel & Gambar
    %-------------------------------------------------------------------------%
    \clearpage
    \addcontentsline{toc}{chapter}{DAFTAR ISI}
    {\tableofcontents}



    \clearpage
    {
        \let\oldnumberline\numberline
        \renewcommand{\numberline}{\figurename~\oldnumberline}
        \listofappendix
    }
    \addcontentsline{toc}{chapter}{DAFTAR LAMPIRAN}

    \clearpage
    {
        \let\oldnumberline\numberline
        \renewcommand{\numberline}{\figurename~\oldnumberline}
        \listoffigures
    }

    \addcontentsline{toc}{chapter}{DAFTAR GAMBAR}

    \clearpage
    {
        \let\oldnumberline\numberline
        \renewcommand{\numberline}{\tablename~\oldnumberline}
        \listoftables
    }

    \addcontentsline{toc}{chapter}{DAFTAR TABEL}

    %-------------------------------------------------------------------------%
    % Konfigurasi Numbering di Bab
    %-------------------------------------------------------------------------%
    \renewcommand{\chaptername}{BAB}
    \renewcommand{\thechapter}{\Roman{chapter}}

    \titleformat*{\section}{\bfseries\large}
    \mainmatter


    % -------------------------------------------------------------------------%
    % Daftar BAB
    % -------------------------------------------------------------------------%
    \chapter{Pendahuluan}


\section{Latar Belakang}

% basically overview TLS, apa itu TLS
\textit{Transport Layer Security} (TLS) merupakan sebuah protokol yang berguna untuk menjamin keamanan dan kerahasiaan dari sebuah pesan yang dikirim melalui jaringan. Saat ini, TLS umum digunakan sebagai sebuah \textit{security layer} dari berbagai protokol yang saat ini umum digunakan di Internet seperti \textit{Hyper-Text Transfer Protocol}, (HTTP) \textit{File Transfer Protocol} (FTP), ataupun \textit{Simple Mail Transfer Protocol} (SMTP). Penggunaan TLS sebagai \textit{layer} tambahan membuat TLS dapat digunakan untuk berbagai protokol dimana keamanan informasi merupakan hal yang penting.

% Apa itu TLS Handshake, masalah TLS handshake sekarang
Sebuah koneksi TLS akan didahului dengan adanya TLS \textit{handshake} untuk menentukan \textit{symmetric key} dan \textit{cipher} yang akan digunakan oleh \textit{client} dan \textit{server} untuk berkomunikasi setelahnya. Sayangnya, waktu yang digunakan untuk TLS \textit{handshake} masih cukup signifikan; penggunaan TLS pada HTTP dapat meningkatkan waktu \textit{load} pada sebuah \textit{website} menjadi 1,2 hingga 3 kali lebih lambat dibandingkan dengan \textit{website} yang tidak menggunakan TLS \citep{cost_of_s}. Hal ini merupakan salah satu alasan kenapa saat ini TLS belum digunakan oleh seluruh layanan yang ada di internet.

% Show bagian mana dari TLS yang bikin lambat, kenalin ke big integer
Penggunaan alogritma kriptografi kunci publik pada dalam TLS Handshake membutuhkan waktu komputasi yang relatif lebih besar dibandingkan dengan proses lain. Waktu yang digunakan oleh komputasi kriptografi kunci publik sendiri adalah 13-59\% dari seluruh proses komputasi pada TLS \citep{perf_tls}. Komputasi kriptografi kunci publik perlu menggunakan bilangan bulat yang besar untuk memastikan kunci yang digunakan dalam kriptografi aman. Komputasi bilangan bulat besar (\textit{big integer}) tersebutlah yang menyebabkan komputasi kriptografi kunci publik  memerlukan waktu yang lama.

% Metode yang disarankan untuk meningkatkan TLS handshake
% Ini perlu nyantumin paper engga?
Terdapat beberapa metode yang telah disarankan untuk meningkatkan kinerja TLS \textit{handshake} seperti melakukan optimisasi pada algoritma Rivest–Shamir–Adleman (RSA) dengan pemrograman paralel, melakukan \textit{batching} pada proses RSA di server, ataupun melakukan \textit{rebalancing} sehingga komputasi pada client lebih sulit daripada server. Solusi lain yang dapat digunakan adalah memparalelkan komputasi \textit{big integer} sehingga komputasi tersebut dapat dilakukan dalam waktu lebih pendek.

% Solusi big integer, berbagai algoritma yang ada dan kemungkinannya untuk diparalelkan
% Montgomery italic ga? nama orang
Komputasi big integer bukanlah hal yang baru dalam dunia ilmu komputer. \citet{modern_comp_math} memberikan tujuh algoritma yang dapat digunakan dalam perkalian, delapan algoritma pembagian, serta berbagai alternatif algoritma  yang dapat digunakan dalam setiap operasi dalam komputasi big integer. Terdapat beberapa juga beberapa algoritma yang dapat dijalankan secara paralel, seperti metode representasi \textit{Residue Number Systems} dan algoritma Montgomery pada komputasi perkalian modular.

Terdapat dua pilihan yang dapat digunakan untuk menjalakan komputasi \textit{big integer} secara paralel, yaitu menjalakan komputasi pada CPU atau GPU. Penggunaan GPU dalam memparalelkan komputasi kriptografi kunci publik dapat menghasilkan jumlah \textit{transaction per second} hingga 45 kali lebih besar dibandingkan dengan penggunaannya pada CPU \citep{gpu_vs_cpu}. Namun, GPU tidak umum digunakan pada data center sehingga penggunaan GPU dalam komputasi kriptografi kunci publik pada kehidupan sehari-hari tidak realistis. Dengan umumnya \textit{multicore} CPU yang digunakan server, penggunaan CPU dalam paralelisasi lebih realistis.


\section{Rumusan Masalah}

Sesuai dengan pembahasan yang ada pada latar belakang, diketahui bahwa penggunaan pemrograman paralel dalam komputasi \textit{big integer} merupakan salah satu metode yang dapat digunakan untuk meningkatkan kinerja TLS \textit{Handshake}. Maka dari itu, tugas akhir ini akan fokus pada pengembangan algoritma paralel yang digunakan dalam komputasi \textit{big integer} dan implementasinya pada TLS \textit{handshake}. Dalam pengembangannya, terdapat berapa masalah yang akan menjadi perhatian utama dalam tugas akhir ini, yaitu:

\begin{enumerate}
  \item Algoritma paralel seperti apa yang akan diimplementasikan dalam komputasi \textit{big integer}?
  \item Bagaimana cara mengintegrasikan algoritma paralel tersebut ke dalam library \textit{big integer}?
  \item Apakah penggunaan algoritma paralel dalam komputasi \textit{big integer} dapat meningkatkan kinerja dari TLS \textit{handshake}?
\end{enumerate}

\section{Tujuan}

Dalam penyelesaian tugas akhir ini terdapat beberapa tujuan yang akan dicapai, yaitu:
\begin{enumerate}
  \item Menemukan algoritma paralel yang dapat digunakan dalam komputasi \textit{big integer}.
  \item Mengintegrasikan algoritma tersebut dalam sebuah implementasi library \textit{big integer}.
  \item Mengukur peningkatan kinerja dari TLS \textit{handshake} setelah mengimplementasikan algoritma paralel RSA.
\end{enumerate}

\section{Batasan Masalah}

Tugas akhir ini akan membahas mengenai optimisasi algoritma RSA pada sebuah implementasi TLS \textit{handshake}. Adapun batasan masalah yang terdapat pada tugas akhir ini adalah:

\begin{enumerate}
  \item Paralelisasi library big integer tidak memperhatikan masalah keamanan pada sistem yang menggunakannya.
  \item Implementasi TLS memanfaatkan perangkat lunak opensource OpenSSL versi 1.1.1.
  \item Ciphersuite yang digunakan dalam penelitian ini adalah TLS\_DH\_RSA\_WITH\_AES\_256\_CBC\_SHA256.
  \item CPU yang digunakan pada implementasi ini adalah Intel 64-bit yang minimal memiliki enam belas \textit{core}.
\end{enumerate}

\section{Metodologi}

Pengerjaan tugas akhir ini akan melalui beberapa tahap, yaitu:
\begin{enumerate}
  \item Analisis Masalah

  Tugas akhir akan dimulai dengan dilakukannya analisis mengenai masalah terhadap implementasi TLS \textit{handshake} yang ada pada saat ini. Pada tahap ini akan dipelajari juga mengenai solusi-solusi untuk meningkatkan kinerja dari TLS \textit{handshake} yang sudah tersedia pada saat ini. Dari solusi yang sudah ada tersebut, penulis akan mencari penyebab solusi itu belum diimplementasikan pada TLS \textit{handshake} dan mencari solusi baru yang mungkin untuk diimplementasikan.

  \item Perancangan Solusi dan Pengujian

  Perancangan solusi dibuat sebagai kerangka dasar atas solusi dari masalah yang telah ditemukan pada langkah sebelumnya. Pada tahap ini akan dirancang arsitektur solusi beserta cara untuk menerapkannya pada TLS handshake yang sudah ada. Selain itu, akan dibuat juga rencana pengujian terhadap implementasi solusi. Rencana pengujian akan dibuat sehingga percobaan dilakukan dalam batas masalah yang ada dan dapat dilakukan sesuai metode sains.

  \item Implementasi Solusi

  Pada tahap ini, penulis akan mengimplementasikan rancangan solusi yang sudah dibuat dalam sebuah bentuk perangkat lunak yang dapat digunakan dalam dunia sehari-hari. Hasil implementasi yang dibuat tentu ditujukan untuk menyelesaikan masalah yang ada.

  \item Pengujian dan Evaluasi

  Pengujian akan dilakukan berdasarkan rancangan pengujian telah ada. Pengujian dilakukan untuk mengetahui apakah solusi yang diberikan merupakan solusi terbaik dari masalah yang ada. Hasil dari pengujian akan digunakan sebagai data evaluasi dan penarikan kesimpulan.

\end{enumerate}


\section{Sistematika Laporan}
\todo[inline]{Diisi nanti kalo laporan udah beres}

% \blindtext
% Penulisan tugas akhir ini akan terdiri dari lima bab yaitu: BAB I Pendahuluan, BAB II Studi Literatur, BAB III Analisis Solusi, BAB IV Rancangan, Implementasi dan Pengujian, BAB V Penutup.
%
% Bab satu memaparkan mengenai latar belakang pembuatan tugas akhir ini, masalah utama yang dibahas, tujuan pembuatan tugas akhir, serta metodologi yang akan dilakukan selama pembuatan tugas akhir. Bab ini dibuat sebagai pendahuluan dan akan mencakup ringkasan dari sebagian besar hal yang dibahas pada tugas akhir.
%
% Bab dua akan membahas mengenai teori dasar yang sudah ada mengenai protokol TLS, algoritma RSA, serta pengembangan dari algoritma RSA melalui pemrograman paralel yang sudah ada. Teori yang diambil akan bersumber dari literatur terpercaya serta dokumentasi resmi dari protokol TLS.
%
% Bab tiga menggambarkan solusi yang digunakan untuk menyelesaikan masalah yang ada, kenapa digunakan solusi tersebut, serta hipotesa peningkatan kinerja dari TLS handshake setelah solusi yang dibuat diimplementasikan.
%
% Bab empat memperlihatkan rancangan implementasi dari solusi, proses implementasi solusi tersebut, serta hasil pengujian dari implementasi tersebut pada kinerja TLS handshake.
%
% Bab lima mendeskripsikan hasil kesimpulan dari solusi yang dibuat untuk menyelesaikan masalah serta saran yang dapat dilakukan untuk pengembangan dan perbaikan agar implementasi ini dapat menghasilkan hasil yang lebih baik.

    %!TEX root = ../../tugas-akhir.tex
\chapter{STUDI LITERATUR}

  % Outline:
  % 2.1 big integer
  % 2.2 Kripto
  % 2.3 TLS
  % 2.4 Parallel Programming
  % 2.5 Other maths Chinese remainder theorem, modulo math, etc
  % 2.6 Relevant Researches

  %!TEX root = ../../../tugas-akhir.tex
\section{Kriptografi}
  Kriptografi adalah sebuah ilmu yang mempelajari cara pengiriman pesan sehingga tidak memungkinkan bagi pihak ketiga untuk membaca ataupun memanipulasi isi pesan tersebut. Kriptografi sudah digunakan sejak jaman kerajaan Romawi Kuno dengan digunakannya Caesar Cipher oleh Julius Caesar untuk berkomunikasi dengan Jendral Perangnya. Kriptografi modern berfokus pada teori matematis dan kesulitan pihak ketiga dalam membaca atau memanipulasi pesan secara komputasional.

  Pada berbagai literatur kriptografi sering dikenal beberapa istilah seperti:
  \begin{enumerate}[label=\roman*.]
    \item \textit{sender}, yaitu pihak pengirim pesan
    \item \textit{receiver}, yaitu pihak penerima pesan
    \item \textit{eavesdropper}, yaitu pihak ketiga yang berusaha membaca atau memodifikasi pesan yang dikirim
    \item \textit{plaintext}, yaitu pesan yang dikirim. \textit{Plaintext} tidak harus merupakan sebuah teks namun dapat berupa audio, gambar, ataupun file binary.
    \item \textit{ciphertext}, yaitu pesan yang telah diubah oleh algoritma tertentu
    \item enkripsi, yaitu proses pengubahan \textit{plaintext} menjadi \textit{ciphertext}
    \item dekripsi, yaitu proses pengubahan \textit{ciphertext} menjadi \textit{plaintext}
    \item \textit{key}, yaitu nilai yang digunakan dalam enkripsi ataupun dekripsi
  \end{enumerate}

  Pada umumnya, proses yang terjadi dalam penggunaan kriptografi pada sebuah pengiriman sebuah data dapat dilihat pada Gambar ~\ref{fig:krypto_system}:

  \begin{figure}[h]
    \centering
    \includegraphics[width=0.6\textwidth]{resources/img/ch-2/crypto-system.jpg}
    \caption{Sistem pengiriman data menggunakan kriptografi}
    \label{fig:krypto_system}
  \end{figure}

  Terdapat dua macam tipe kriptografi yang saat ini umum digunakan. Kriptografi kunci simetri adalah sebuah sistem kriptografi dimana proses enkripsi dan dekripsi menggunakan kunci yang sama. Kriptografi asimetris atau biasa disebut kriptografi kunci publik merupakan sebuah sistem kriptografi dimana proses enkripsi dan dekripsi menggunakan dua kunci yang berbeda.

  Masalah utama yang dihadapi pada sistem kriptografi kunci simetri adalah bagaimana dua pihak dapat menentukan sebuah kunci yang akan digunakan tanpa sepengetahuan pihak ketiga. Pada algoritma kriptografi modern, terdapat beberapa metode untuk mengirimkan kunci simetri pada pihak lain, diantaranya adalah mendekripsi key tersebut dengan algoritma kriptografi kunci publik atau dengan menggunakan algoritma \textit{key exchange}.

  \subsection{Kriptografi Kunci Publik}

    Kriptografi kunci publik adalah sebuah sistem kriptografi dimana proses enkripsi dan dekripsi dilakukan dengan kunci yang berbeda. Umumnya terdapat dua macam kunci yang terdapat pada sistem ini, yaitu \textit{public key} yang dapat dipublikasikan kepada siapapun dan \textit{private key} yang hanya dimiliki oleh pemilik kunci tersebut. \textit{Public key} digunakan untuk mengenkripsi data, sementara \textit{private key} digunakan oleh pemilik kunci untuk mendekripsi data tersebut.

    Salah satu kebutuhan utama yang terdapat pada kriptografi kunci publik adalah sebuah algoritma yang mudah untuk dikomputasi dalam satu arah, namun sulit untuk dilakukan sebaliknya. Algoritma yang digunakan umumnya berdasar pada masalah matematis seperti faktorisasi bilangan bulat, logaritma diskrit, atau hubungan kurva eliptik. Sebagai contoh, jika kita memiliki dua bilangan bulat maka dapat dengan mudah mengkalikan dua bilangan tesebut dan mendapat satu bilangan bulat baru; namun kita tidak bisa dengan mudah menentukan dua bilangan bulat yang merupakan faktor dari sebuah bilangan bulat.

    Saat ini komputasi kriptografi kunci publik masih memerlukan proses komputasi yang lebih tinggi dibandingkan dengan kriptografi kunci simetris. Maka dari itu, kriptografi kunci publik biasanya hanya dilakukan untuk mengenkripsikan kunci yang akan digunakan dalam kriptografi kunci simetris.

    \subsubsection{Rivest–Shamir–Adleman (RSA)} \label{sec:rsa_theory}
    \todo{edit lagi, bahasanya ngaco}
    RSA merupakan salah satu algoritma kriptografi kunci publik yang hingga saat ini banyak digunakan. Algoritma RSA pertama kali dicetuskan oleh Ron Rivest, Adi Shamir, dan Leonard Adleman pada tahun 1978. RSA berdasar pada sifat matematis bahwa dalam sebuah pemangkatan modular dengan persamaan:
    \begin{equation}
      (m^e)^d  \equiv  m \pmod{n}
    \end{equation}

    Pencarian tiga bilangan bulat n, e, dan d dapat dilakukan dengan mudah; bahkan untuk bilangan n, e, dan d yang sangat besar. Namun, pencarian bilangan d akan susah dilakukan, bahkan jika kita mengetahui bilangan-bilangan lainnya. Selain itu mengingat operasi perpangkatan dapat ditukar, maka proses enkripsi dan dekripsi dapat dilakukan dengan metode yang sama, hanya menggunakan bilangan yang berbeda.

    Terdapat beberapa proses yang terjadi pada algoritma RSA yaitu pembuatan \textit{public} dan \textit{private key}, distribusi \textit{public key}, enkripsi, serta dekripsi. Proses pembuatan public dan private key sendiri dilakukan dalam beberapa tahap yaitu:

    \begin{enumerate}
      \item Pilih dua bilangan prima yang besar, $p$ dan $q$.
      \item Hitung $n = p*q$ dan $ \phi = (p-1)*(q-1)$ .
      \item Pilih sebuah bilangan bulat random $e$ dengan $ 1 < e < \phi$ dan .
      \item Hitung bilangan bulat $d$, dengan $ e*d  \equiv  1 \pmod{\phi} $
    \end{enumerate}

    Dari perhitungan diatas, didapat \textit{public key} $(e, n)$ dan private key $(d, n)$.
    Proses enkripsi pada sebuah data \textit{D} dapat dilakukan melalui beberapa langkah sebagai berikut:
    \begin{enumerate}
      \item Ubah \textit{D} menjadi satu atau beberapa bilangan bulat \textit{m}, dengan \textit{m} berada dalam interval [1..n-1]
      \item Hitung ciphertext \textit{c} untuk masing-masing \textit{m}, dengan $c = m^e \mod n $
    \end{enumerate}

    Sementara itu, proses dekripsi dapat dilakukan dengan menghitung plaintext $m$ dari ciphertext $c$ yang didapatkan dengan $m = c^d \mod n$.


  \subsection{Algoritma Pertukaran Kunci}
  Algoritma pertukaran kunci adalah algoritma yang digunakan untuk menghasilkan sebuah kunci rahasia antara dua atau lebih pihak tanpa diketahui oleh pihak lain \citep{applied_crypto}. Kunci yang dihasilkan dari algortima ini akan digunakan sebagai kunci pada sistem kriptografi simetris.

    \subsubsection{Diffie-Hellman}
    Algoritma Diffie-Hellman (DH) merupakan salah satu algoritma paling awal yang memperkenalkan konsep pertukaran kunci. Walaupun DH merupakan algoritma awal yang ada, DH masih umum digunakan pada sistem kriptografi modern. Kriptografi modern hanya menggunakan modifikasi dari algoritma DH, misalnya dengan menggunakan kurva eliptik sebagai basis perhitungannnya.

    Seperti RSA, DH memiliki basis perpangkatan modulo bilangan bulat. Pada awal algortima DH, dipilih dua nilai publik $p$ dan $g$, dengan $p$ merupakan bilangan prima besar dan $g$ adalah akar primitif modulo dari $p$. Setelah itu, masing-masing pihak akan memilih nilai random $x$ dengan $1 \leq x \leq p-1 $. Setelah nilai random dipilih akan dihitung nilai $pk = g^x \mod p$ pada masing-masing pihak, kemudian nilai $pk$ akan dikirim melalui jaringan pada pihak yang lain. Setelah nilai dari pihak lain diterima, akan dihitung nilai $key = prekey^x \mod p$. Ilustrasi pertukaran kunci dapat dilihat pada Gambar \ref{fig:dh_exchange}.

    \begin{figure}[h]
      \centering
      \includegraphics[width=0.9\textwidth]{resources/img/ch-2/dh-exchange}
      \caption{Proses pertukaran kunci Diffie-Hellman}
      \label{fig:dh_exchange}
    \end{figure}

    Meskipun nilai $p, g, $ serta $pk$ merupakan nilai publik yang dapat diketahui semua orang, pihak ketiga tidak dapat menghitung nilai akhir $key$ yang didapatkan dari perhitungan pada masing-masing server. Untuk mendapatkan $key$ yang digunakan oleh dua pihak, dibutuhkan nilai random $x$ yang dipilih oleh masing-masing pihak tersebut. Perhitungan nilai $x$ dari nilai $g^x mod p$ merupakan masalah logartima diskrit, yaitu masalah yang sama dengan pencarian kunci privat dari RSA.

  \section{\textit{Transport Layer Security} (TLS)}
  \textit{Transport Layer Security} (TLS) adalah sebuah protokol kriptografi yang menjamin kerahasiaan serta integritas dalam data yang dikirim dalam sebuah koneksi yang tidak dipercaya. TLS pertama kali dikembangkan sebagai \textit{Secure Socket Layer} (SSL) oleh Netscape pada tahun 1995, dan kemudian diresmikan oleh Internet Engineering Task Force (IETF) dalam RFC 2246 pada tahun 1998. SSL dirancang dengan empat tujuan utama yaitu keamanan secara kriptografis, interoperabilitas, kemampuan penambahan protokol, serta efisiensi \citep{perf_tls}.

  TLS dirancang sebagai sebuah protokol perantara antara protokol aplikasi (HTTP, FTP, SMTP) dan protokol transport (TCP, UDP). Semua data yang dikirimkan oleh aplikasi akan dienkripsi terlebih dahulu oleh TLS sehingga data dapat dikirimkan secara terenkripsi. Hal ini dapat mempermudah penggunaan TLS diatas protokol aplikasi mengingat aplikasi hanya perlu mengirimkan data melalui TLS socket alih-alih melalui TCP/UDP socket untuk kemudian dienkripsi dan dikirimkan pada tujuan melalui TCP/UDP socket. Ilustrasi perbandingan antara pengiriman dengan menggunakan TLS dan tidak menggunakan TLS dapat dilihat pada Gambar ~\ref{fig:tls-usage}

  \begin{figure}[h]
    \centering
    \includegraphics[width=0.6\textwidth]{resources/ch-2/tls-usage.png}
    \caption{Perbandingan Penggunaan Protokol TLS pada Aplikasi \protect\citep{perf_tls}}
    \label{fig:tls-usage}
  \end{figure}

  Dewasa ini, TLS umum digunakan sebagai layer keamanan pada protokol aplikasi yang umum digunakan di internet. Berbagai protokol yang awalnya tidak memiliki standar keamanan dapat menggunakan TLS untuk mengamankan aliran data. Sebagai contoh, penggunaan HTTPS dapat memastikan bahwa data-data penting seperti \textit{password}, nomor kartu kredit, ataupun alamat tidak dapat dilihat oleh siapapun kecuali pengguna dan pemilik dari sebuah website.

  TLS merupakan sebuah protokol asimetrik dimana pihak yang melakukan koneksi dapat dibagi menjadi client dan server. TLS menyediakan metode otentikasi sehingga sebuah pihak dapat yakin bahwa ia memang berkomunikasi dengan pihak yang ia inginkan. TLS menyediakan metode otentikasi baik untuk client maupun server, walaupun pada praktiknya biasanya hanya server yang diotentikasi oleh client. Salah satu metode lain adalah TLS memastikan bahwa pesan yang dikirim tidak diubah pada perjalanan dengan adanya penambahan \textit{Message Authentication Code} (MAC) pada semua data yang dikirim.

  Protokol TLS pada dasarnya tersusun dari dua protokol utama, yaitu Protokol TLS \textit{Record} dan Protokol TLS \textit{Handshake}. Protokol TLS Record digunakan untuk mengenkapsulasi data yang diterima oleh protokol aplikasi, sementara itu Protokol TLS Handshake digunakan untuk mengotentikasi client dan server serta melakukan negosiasi mengenai pemilihan penggunaan algoritma enkripsi, kunci yang digunakan, serta pemilihan penggunaan algoritma MAC (Message Authentication Code).

% Pengembang dari TLS sadar bahwa komputasi kriptografis yang digunakan dalam otentikasi server memerlukan biaya komputasi yang cukup tinggi. Karena itulah, TLS memiliki sebuah mekanisme session yang memungkinkan client yang telah menyelesaikan TLS \textit{Handshake} sebelumnya untuk melakukan TLS Handshake secara cepat dengan menggunakan ulang data session yang masih tersimpan pada client. Pada sebuah session, data yang tersimpan diantaranya:
% \begin{description}
%   \item[-] \textit{Session identifier}
%   \item[-] \textit{Peer cerficate}
%   \item[-] \textit{Compression method}
%   \item[-] \textit{Cipher spec}, yaitu pasangan algoritma yang digunakan dalam enkripsi dan MAC.
%   \item[-] \textit{Master secret}, yaitu key 48-byte yang digunakan dalam enkripsi dan dekripsi.
%   \item[-] \textit{is resumable}, menandakan apakah session ini dapat digunakan untuk membuat session baru.
% \end{description}


  \subsection{Protokol TLS \textit{Handshake}}
    TLS \textit{Handshake} adalah protokol yang digunakan untuk melakukan otentikasi serta negosiasi untuk menentukan parameter yang digunakan pada sebuah koneksi TLS. Sebuah TLS \textit{handshake} juga digunakan untuk membuat dan menentukan \textit{session} dari sebuah koneksi TLS. Protokol ini juga mendeskripsikan cara yang digunakan oleh TLS untuk memberikan tanda pada satu sama lain jika terjadi error pada salah satu tahap yang terjadi.

    Tahap yang terjadi dalam sebuah TLS \textit{Handshake} dapat dideskripsikan sebagai berikut:
  \begin{enumerate}
    \item Pengiriman ‘Hello’ sebagai tanda mulainya koneksi.
    \item Pengiriman algoritma yang didukung, nilai random, serta pengecekan untuk penggunaan kembali session.
    \item Pengiriman parameter yang digunakan untuk melakukan komputasi kunci simetris.
    Pengiriman sertifikat oleh server dan otentikasi server oleh client.
    \item Melakukan komputasi kunci simetris secara independen, kemudian menyimpan parameter-parameter yang dibutuhkan untuk penyimpanan data.
    \item Melakukan verifikasi bahwa semua pihak telah menghitung kunci simetris yang sama dan memastikan proses handshake tidak diganggu oleh siapapun.
  \end{enumerate}

    Apabila terdapat \textit{error} pada salah satu tahap diatas, maka akan dikirimkan tanda \textit{error} dan pembuatan koneksi dihentikan seketika. Proses pengiriman data yang terjadi pada sebuah TLS Handshake dapat dilihat pada Gambar ~\ref{fig:tls-handshake}.

    \begin{figure}[h]
      \centering
      \includegraphics[width=0.6\textwidth]{resources/ch-2/handshake.png}
      \caption{Proses Pengiriman Data pada TLS Handshake \protect\citep{rfc5246}}
      \label{fig:tls-handshake}
    \end{figure}

    Pada Gambar ~\ref{fig:tls-handshake}, terlihat bahwa proses handshake merupakan proses yang memakan waktu cukup signifikan. Hal ini disebabkan karena terdapat setidaknya empat kali pengiriman data antara client dan server, selain itu proses \textit{key exchange} dan verifikasi sertifikat yang dilakukan tidak memakan waktu yang singkat.

    Untuk mengatasi hal ini, TLS menyediakan proses handshake singkat dimana client akan menggunakan \textit{session\_id} yang dimilikinya untuk melanjutkan session tersebut. Pada proses ini, client akan mengirimkan data \textit{session\_id} bersama ClientHello; server kemudian akan mencari data session pada session cache dan melanjutkannya jika ditemukan. Jika proses ini gagal, maka proses handshake akan dilakukan seperti biasa. Proses pengiriman data yang dilakukan pada handshake singkat dapat dilihat pada Gambar ~\ref{fig:tls-fast-handshake}.

    \begin{figure}[h]
      \centering
      \includegraphics[width=0.6\textwidth]{resources/ch-2/fast-handshake.png}
      \caption{Proses Pengiriman Data pada TLS Handshake Singkat \protect\citep{rfc5246}}
      \label{fig:tls-fast-handshake}
    \end{figure}

    Terlihat bahwa penggunaan ulang \textit{session} dapat mempersingkat waktu yang digunakan dalam TLS \textit{handshake} secara signifikan mengingat tidak diperlukannya lagi validasi sertifikat ataupun proses \textit{key exchange}.

  \subsection{Protokol TLS \textit{Record}}
    Protokol TLS \textit{Record} merupakan protokol pemrosesan data yang didapat dari aplikasi untuk kemudian dikirim melalui jaringan dan juga sebaliknya. Ketika TLS menerima data dari sebuah aplikasi, data tersebut akan dibagi menjadi beberapa blok data untuk mempermudah pengolahan, data kemudian akan di-\textit{compress}, ditambahkan MAC, dienkripsi, lalu kemudian dikirim melalui jaringan. Sebaliknya, ketika TLS menerima data dari jaringan, data akan didekripsi, diverifikasi dengan MAC, di-decompress, disusun ulang, lalu dikirim pada aplikasi.

    Pada pengiriman data, TLS menggunakan format data tertentu seperti diilustrasikan pada Gambar ~\ref{fig:tls-record}:
    \begin{figure}[h]
      \centering
      \includegraphics[width=0.6\textwidth]{resources/ch-2/tls-record.png}
      \caption{Struktur Sebuah Record TLS \protect\citep{rfc5246}}
      \label{fig:tls-record}
    \end{figure}

    % Untuk setiap koneksi yang ada, TLS akan beberapa nilai yang menggambarkan status dari koneksi tersebut. Salah satu data yang disimpan adalah algoritma kompresi, enkripsi, serta MAC yang digunakan. Selain itu, berbagai parameter yang diperlukan oleh setiap algoritma seperti kunci yang digunakan untuk enkripsi dan MAC juga akan disimpan.

    Proses enkripsi yang terjadi pada tahap ini akan menggunakan sebuah algoritma kriptografi kunci simetris seperti AES ataupun RC4 dengan kunci yang digunakan dalam algoritma tersebut didapatkan oleh client dan server pada TLS Handshake. Penggunaan kriptografi kunci simetris akan membuat proses enkripsi dan dekripsi menjadi relatif lebih cepat jika dibandingkan dengan penggunaan algoritma kriptografi kunci publik.


  \subsection{Ciphersuite}
    \textit{Ciphersuite} adalah sebuah kumpulan dari algoritma yang digunakan dalam TLS \textit{handshake}. Sebuah \textit{ciphersuite} akan terdiri dari algoritma tertentu yang digunakan dalam sebuah koneksi TLS. Kumpulan algoritma tersebut biasanya akan berisi algoritma otentikasi, algoritma \textit{key exchange}, algoritma enkripsi data pada TLS record, serta algoritma MAC. Terdapat banyak kombinasi dari ciphersuite yang umum digunakan pada protokol TLS, beberapa diantaranya memiliki tingkat keamanan yang lebih rendah dibandingkan dengan ciphersuite lain.

    Setiap TLS \textit{client} dan \textit{server} akan memiliki daftar \textit{ciphersuite} yang didukung. Daftar tersebut akan mengurutkan setiap ciphersuite dengan urutan tertentu berdasarkan preferensi dari masing-masing client dan server. Urutan ini biasanya disusun berdasarkan tingkat keamanan dari masing-masing \textit{ciphersuite}. \textit{Ciphersuite} yang memiliki tingkat keamanan tinggi akan memiliki urutan yang lebih prioritas dibandingkan \textit{ciphersuite} dengan tingkat keamanan yang lebih rendah.

    Sebuah ciphersuite biasanya memiliki format penulisan tertentu, sebagai contoh TLS\_DHE\_ RSA\_WITH\_AES\_256\_CBC\_SHA256 adalah \textit{ciphersuite} dengan algoritma key exchange \textit{ephemeral Diffie-Hellman}, algoritma enkripsi simetris AES256, serta SHA256 yang digunakan sebagai algoritma MAC.

    Proses negosisasi ciphersuite antara client dan server dilakukan dengan sederhana. Pertama \textit{ciphersuite} akan dikirim pada pesan ClientHello dan ServerHello. Ketika setiap pihak sudah mendapatkan daftar \textit{ciphersuite} rekannya, ia akan mencari \textit{ciphersuite} tertinggi pada daftar preferensi rekannya dimana ia juga mendukung \textit{ciphersuite} tersebut.

  %!TEX root = ../../../tugas-akhir.tex
\section{Big Integer}
  % Arithmetic di komputer
  Komputasi matematis di dalam sebuah komputer dilakukan oleh Arithmetic Logic Unit (ALU) yang terdapat di dalam CPU. ALU merupakan sebuah komponen yang sangat sederhana, karena itu ALU memiliki banyak batasan. Salah satunya adalah ALU hanya dapat beroperasi pada bilangan bulat dalam \textit{range} tertentu \citep{comp_org_arch}. Pada umumnya, sebuah CPU memiliki 32-bit atau 64-bit ALU, nilai maksimum bilangan bulat yang dapat diproses oleh ALU tersebut hanya sebesar $2^{64}$. Kemampuan ALU untuk menangani bilangan bulat pada \textit{range} tertentu berdasarkan jumlah bit yang dimilikinya biasa dikenal sebagai \textit{fixed-precision integer arithmetic}.

  % Big number, Apa itu, kenapa dibutuhkan, dikenal juga sebagai Arbitrary/multi precision
  Untuk menangani operasi matematis yang menggunakan bilangan yang lebih besar dari \textit{range} yang dimiliki ALU, diperlukan sebuah struktur data yang dapat menangani bilangan bulat tersebut. Kemampuan komputer untuk menghitung bilangan yang tidak memiliki batas sering dikenal sebagai \textit{arbitrary-precision integer arithmetic}. Sementara itu, bilangan yang digunakan dalam perhitungan tersebut sering disebut \textit{big number} atau \textit{big integer} jika bilangan yang digunakan adalah bilangan bulat.

  % dipake dimana aja
  \textit{Big integer} sering digunakan pada perhitungan kriptografi, mengingat bahwa perhitungan kriptografi membutuhkan bilangan yang besar agar kunci yang digunakan aman. Sebagai contoh, disarankan untuk menggunakan kunci sebesar 256bit untuk AES dan kunci sebesar 2048 bit untuk RSA \citep{key_suggestion} agar enkripsi yang digunakan aman. Jumlah bit yang digunakan tersebut lebih besar dari jumlah bit yang dapat ditangani oleh ALU. Selain perhitungan kriptografi, big number juga umum digunakan untuk menghitung nilai konstanta matematis seperti $\pi$ \citep{bn_pi}, .


  \subsection{Representasi Big Integer}

      \citet{modern_comp_math} menyatakan bahwa sebuah bilangan bulat dapat direpresentasikan sebagai penjumlahan dari komponen-komponennya. Jika kita memilih sebuah bilangan bulat positif $\beta$ sebagai basis dengan $\beta > 1 $ semua bilangan bulat positif $A$ yang memiliki basis $\beta$ dengan memiliki panjang $n$ dapat dituliskan sebagai:
      \begin{equation} \label{eq:frns_rep}
        A = a_{n-1}\beta^{n-1}+a_{n-2}\beta^{n-2}+...+a_{1}\beta+a_{0}
      \end{equation}
      dengan $0 \leq \alpha \leq \beta -1$.

      Selain penulisan pada persamaan \ref{eq:frns_rep}, A juga dapat ditulis dalam notasi basis-n seperti $A = (a_{n-1} a_{n-2} ... a_{1} a_{0})_\beta$.

      Representasi integer positif di komputer 64 bit menggunakan $\beta = 2$ dan $n = 64$ sehingga nilai maksimum yang dapat direpresentasikan adalah $2^{64}$. Untuk representasi big integer nilai $n$ tidak memiliki batas, sementara nilai $\beta$ yang digunakan sesuai dengan jumlah maksimum bilangan dapat diproses pada komputer tersebut. Pada representasi big integer di komputer 64 bit, digunakan $\beta = 2^{64}$.

      % Physical yang umum, array
      % masuk bab 3?
      Representasi ideal big integer pada komputer adalah menggunakan list of integer atau array of integer. Penggunaan list menyebabkan big integer tidak memiliki nilai maksimum, sementara itu penggunaan array membuat akses nilai yang tersimpan lebih cepat. Penggunaan array of integer lebih umum digunakan untuk merepresentasikan big integer. Ilustrasi array of integer yang merepresentasikan big integer dapat dilihat pada Gambar \ref{fig:frns_ref}.

      \begin{figure}[h]
        \centering
        \includegraphics[width=0.5\textwidth]{resources/img/ch-2/frns-ref.png}
        \caption{Representasi FRNS dalam bentuk Array of Integer}
        \label{fig:frns_ref}
      \end{figure}

      Gambar \ref{fig:frns_ref} memberikan gambaran bahwa untuk merepresentasikan big integer $A = (a_{n-1} a_{n-2} ... a_{1} a_{0})_\beta$, dapat digunakan array $A = [0..n-1]$, dengan nilai $\beta$ dapat disimpan pada sebuah variabel terpisah. Sebagai contoh, array $N = [2^{29}, 3^{31}, 2^{54}, 3^{27}]$ merupakan representasi dari big integer $N = 3^{27}*2^{192} + 2^{54}*2^{128} + 3^{31}*2^{64} + 2^{29}$.

  \subsection{Operasi Aritmatika}

    \subsubsection{Penjumlahan dan Penguragan} \label{sec:add_sub_theory}

      Operasi penjumlahan dan pengurangan merupakan algoritma yang biasa digunakan dalam melakukan penjumlahan dan pengurangan yang diajarkan di sekolah dasar. Merujuk pada persamaan \ref{eq:frns_rep}, proses penjumlahan dan pengurangan dilakukan dari $\alpha_0$ hingga $\alpha_{n-1}$ dengan menggunakan bilangan \textit{carry over} atau \textit{borrow} jika dibutuhkan.

      Untuk penjumlahan dan pengurangan dua bilangan $A = (a_{n-1} a_{n-2} ... a_{1} a_{0})_\beta$ dan $B = (b_{n-1} b_{n-2} ... b_{1} b_{0})_\beta$, akan dihasilkan bilangan $C = (c_{n-1} c_{n-2} ... c_{1} c_{0})_\beta$. Pseudocode \ref{alg:add} adalah algoritma yang dapat digunakan untuk menghitung $C$ dari Penjumlahan $A$ dan $B$.

      \begin{algorithm}
        \caption{Algoritma Penjumlahan}
        \label{alg:add}
        \begin{algorithmic}[1]
          \Require{$A$ dan $B$ adalah array [1..n] yang merepresentasikan Big Integer, $\beta$ adalah basis yang digunakan}
          \Statex
          \Function{Add}{$A$, $A$, $n$}
          \Let{$C$}{\Call{NewArray}{$n$}}
          \Let{$carry$}{0}
          \For{$i \gets 1$, $i \gets i + 1$ to $n$}
          \Let{$n$}{$carry$ + $A[i]$ + $B[i]$}
          \Let{$carry$}{$n \div \beta$}
          \Let{$C[i]$}{$n \mod \beta$}
          \EndFor
          \Let{$C[n+1]$}{$carry$}
          \State \Return{$z$}
          \EndFunction
        \end{algorithmic}
      \end{algorithm}

      Algoritma yang digunakan dalam pengurangan hampir sama dengan penggunaan algoritma \ref{alg:add}. Hal yang berbeda hanyalah pada baris ke-5 dengan $n = x[i] - y[i] - carry $

    \subsubsection{Perkalian} \label{sec:mul_theory}

      Terdapat beberapa algoritma yang dapat dilakukan dalam perkalian dua bilangan besar. Pada subbab ini akan dibahas tiga algoritma perkalian, yaitu algoritma perkalian panjang, algoritma perkalian comba dan algoritma perkalian Karatsuba. Ketiga algoritma tersebut merupakan algoritma yang digunakan oleh OpenSSL.

      Algoritma perkalian panjang adalah algoritma perkalian yang biasa diajarkan dalam sekolah dasar. Algoritma perkalian panjang $A*B$ berdasarkan pada proses mengkalikan satu digit A pada B, geser hasilnya sesuai posisi digit A yang digunakan, lalu jumlahkan seluruh hasil perkalian tersebut. Algoritma ini memiliki kompeksitas $O(n^2)$ terhadap perkalian. Pseudocode \ref{alg:mul} merupakan gambaran dari algoritma perkalian panjang.

      \begin{algorithm}
        \caption{Algoritma Perkalian Panjang}
        \label{alg:mul}
        \begin{algorithmic}[1]
          \Require{$A$ = [1..p], $B$ = [1..q] yang merepresentasikan Big Integer, $\beta$ adalah basis yang digunakan representasi Big Integer}
          \Statex
          \Function{Mul}{$A$, $B$, $\beta$}
          \Let{C}{\Call{NewArray}{p+q}}
          \For{$i \gets 1$, $i \gets i + 1$ to $q$}
          \Let{$carry$}{0}
          \For{$j \gets 1$, $j \gets j + 1$ to $q$}
          \Let{$C[i+j-1]$}{$C[i+j-1] + carry + A[i] * A[j]$}
          \Let{$carry$}{$C[i+j-1] \div \beta$}
          \Let{$C[i+j-1]$}{$C[i+j-1] \mod \beta$}
          \EndFor
          \Let{$C[j+p]$}{$carry$}
          \EndFor
          \State \Return{$C$}
          \EndFunction
        \end{algorithmic}
      \end{algorithm}

      Algoritma Comba merupakan modifikasi dari algoritma perkalian panjang. Berbeda dengan algoritma perkalian panjang, algoritma comba melakukan proses iterasi perkalian berbasis kolom, sementara algoritma perkalian panjang berbasis baris. Penggunaan kolom dibandingkan baris membuat carry pada setiap iterasi perkalian lebih sedikit. Dengan demikian, algoritma comba mengurangi kompleksitas ruang serta mengurangi jumlah penulisan terhadap memori.

      Algoritma karatsuba berdasarkan bahwa perkalian dua bilangan A dan B dapat direpresentasikan dalam tiga perkalian bilangan bulat yang lebih kecil. Algoritma ini memiliki kompeksitas $O(n^{\log_2 3})$ terhadap operasi perkalian. Penggunaan algoritma karatsuba dalam komputer dapat dilakukan dengan pemanggilan fungsi secara rekursif. Algoritma karatsuba dalam pseudocode dapat dilihat pada Pseudocode \ref{alg:karatsuba_mul}

      \begin{algorithm}
        \caption{Algoritma Perkalian Karatsuba}
        \label{alg:karatsuba_mul}
        \begin{algorithmic}[1]
          \Require{$A$ = [1..p], $B$ = [1..q] yang merepresentasikan Big Integer, $\beta$ adalah basis yang digunakan}
          \Statex
          \Function{MulKaratsuba}{$A$, $B$, $\beta$}
          \If{($p = 1$) or ($q = 1$)}
          \State \Return $A * B$
          \EndIf
          \Let{$mid$}{\Call{Floor}{\Call{Min}{$p, q$}/2}}
          \Let{$A_{low}, A_{high}$}{\Call{SplitIn}{$A, mid$}}
          \Comment{Split $A$ into two subarray at $mid$}
          \Let{$B_{low}, B_{high}$}{\Call{SplitIn}{$B, mid$}}
          \Let{$C_0$}{\Call{MulKaratsuba}{$A_{low}, B_{low}$}}
          \Let{$C_1$}{\Call{MulKaratsuba}{($A_{low} + A_{high}$), ($B_{low} + B_{high}$)}
          }
          \Let{$C_2$}{\Call{MulKaratsuba}{$A_{high}, B_{high}$}}
          \State
          \Let{$x$}{$C_2 * \beta ^ {mid * 2}$}
          \Let{$y$}{$(C_1 - C_2 - C_0) * \beta ^ {mid}$}

          \State \Return{$x + y + C_1$}
          \EndFunction
        \end{algorithmic}
      \end{algorithm}

    \subsubsection{Pembagian}\label{sec:div_theory}

      Seperti perkalian, terdapat beberapa algoritma yang dapat digunakan dalam pembagian. Seperti operasi aritmatika sebelumnya, algoritma pembagian yang banyak digunakan adalah algoritma pembagian panjang yang diajarkan pada sekolah dasar. Pada implementasinya di komputer, algoritma ini memiliki prekondisi divisor yang telah dinormalisasi. Divisor yang dinormalisasi adalah divisor $D = [0..n-1]$ yang menggunakan basis $\beta$, berlaku $D[n-1] > \beta/2$. Pseudocode \ref{alg:long_div} menjelaskan algoritma ini dalam bentuk pseudocode.

      \begin{algorithm}
        \caption{Algoritma Pembagian Panjang}
        \label{alg:long_div}
        \begin{algorithmic}[1]
          \Require{$N = [0..n+m-1], D = [0..n-1]$ bilangan bulat dalam representasi array.}
          \Statex
          \Function{LongDiv}{$D, N, \beta$}
              \Let{$Q$}{\Call{NewArray}{m}}
              \Let{$is_bigger$}{\Call{BigIntCompare}{$N$, \Call{Mul}{$D, \beta^m$}}}
              \If{$is_bigger$}
                \Let{$Q[m]$}{1}
              \Else
                \Let{$Q[m]$}{0}
              \EndIf
              \For{$i \gets m$, $i \gets i - 1$ downto $n$}
                  \Let{$Q[i]$}{$\lfloor\frac{N[n+i]\beta + N[n+j-1]}{D[n-1]}\rfloor$}
                  \Let{$Q[i]$}{\Call{Min}{$Q[i],\beta$}}
                  \Let{$N$}{$N$-\Call{Mul}{$Q[i], D}*\beta^i$}
                  \While{$A$ < 0}
                      \Let{$Q[i]$}{$Q[i] - 1$}
                      \Let{$N$}{$N+N\beta^i$}
                  \EndWhile
                  \Let{$R$}{$A$}
              \EndFor
              \State \Return{$Q, R$}
          \EndFunction
        \end{algorithmic}

      \end{algorithm}

      % \citet{div_burnikel_ziegler} memperkenalkan algoritma pembagian cepat yang berjalan pada kompleksitas $2K(n)+O(n \log n)$ dengan $K(n)$ adalah waktu berjalannya algoritma perkalian karatsuba.
    \subsubsection{Perpangkatan}
        Perpangkatan merupakan operasi yang membutuhkan waktu komputasi paling tinggi dibandingkan dengan operasi aritmatika dasar yang lain. Algoritma naif adalah melakukan perkalian secara berulang sebanyak jumlah perpangkatan. Namun, terdapat beberapa algoritma perpangkatan yang dapat digunakan untuk melakukan perpangkatan secara lebih efektif.
        \citet{exp_method} menjelaskan tiga algoritma perpangkatan $r=a^e$; yaitu metode binary, metode \textit{m}-ary, dan metode sliding window.

        Perpangkatan binary yang biasa juga disebut \textit{square and multiply} merupakan metode yang berbasis pada penulisan eksponen $e$ dalam basis binary $e=(e_{n-1} e_{n-2} ... e_{1} e_{0})_2$. Algoritma akan melakukan iterasi baik dari digit terbesar hingga digit terkecil; melakukan perkalian jika pada $a$ jika $e_{i} = 1$, dan melakukan perpangkatan dua ketika berpindah ke $e_{i}$ selanjutnya. Pseudocode \ref{alg:binary_exp} menggambarkan pseudocode untuk algoritma ini.

        \begin{algorithm}
          \caption{Algoritma Perpangkatan Binary}
          \label{alg:binary_exp}
          \begin{algorithmic}[1]
            \Require{$a = [0..m],$ bilangan bulat dalam representasi array. $e = [0..n-1]$ bilangan bulat dalam basis binary}
            \Statex

              \Function{BinExp}{$A, E$}
            \Let{$r$}{$1$}
            \For {$i \gets n-1$, $i \gets i - 1$ downto $0$}
                \Let{$r$}{$r*r$}
                \If {$e[i] = 1$}
                    \Let{$r$}{$r*a$}
                \EndIf
            \EndFor
            \State \Return{$r$}
            \EndFunction
          \end{algorithmic}
        \end{algorithm}

        Algoritma perpangkatan binary dapat digeneralisir lebih jauh dengan menuliskan $e$ pada basis apapun. Penulisan $e$ sebagai $e=(e_{n-1} e_{n-2} ... e_{1} e_{0})_n$ merubah baris ke-4 pada Pseudocode \ref{alg:binary_exp} menjadi $r \gets r^{n}$ dan baris ke-5 dan 6 menjadi $r \gets r*a^{E[i]}$. Generalisasi pada algoritma binary ini merupakan algoritma \textit{m}-ary.

        Algoritma sliding window merupakan modifikasi dari algoritma binary dan algoritma \textit{m}-ary dengan menuliskan $e$ sebagai binary namun melakukan operasi pada beberapa bit sekaligus. Sebagai contoh, E =  17647 dapat dituliskan dalam binary sebagai 100010011101111. Algoritma perpangkatan window akan melakukan partisi E menjadi 1\underline{000}1\underline{00}111\underline{0}1111. Proses perpangkatan kemudian akan dilakukan pada masing-masing elemen window. Untuk mempercepat komputasi, dapat dilakukan prekomputasi elemen window non-nol yaitu $A^{(1)_2}$, $A^{(11)_2}$ $A^{(111)_2}$, dan $A^{(1111)_2}$. Pseudocode \ref{alg:window_exp} merupakan pseudocode yang menggambarkan algoritma perpangkatan window.

        \begin{algorithm}
          \caption{Algoritma Perpangkatan Sliding Window}
          \label{alg:window_exp}
          \begin{algorithmic}[1]
            \Require{$A = [0..m],$ bilangan bulat dalam representasi array. $E = [0..n-1]$ bilangan bulat dalam basis binary}
            \Statex

            \Function{WindowExp}{$A, E$}
            \Let{$w$}{\Call{Partition}{$E$}}
            \Let{$r$}{$1$}
            \For {$i \gets 0$, $i \gets i + 1$ to $length(w)$}
                \Let{$r$}{$r^{2^{length(w[i])}}$}
                \Let{$r$}{$r*a^{w[i]}$}
            \EndFor
            \State \Return{$r$}
            \EndFunction
          \end{algorithmic}
        \end{algorithm}

    \subsubsection{Perkalian Modular} \label{sec:modmul}
    % jelasin perkalian dasar a*b mod n = (a mod n * b mod n) mod n
    Operasi perkalian modular merupakan operasi yang lebih kompleks dibandingkan dengan operasi perkalian biasa. Metode paling sederhana untuk melakukan perkalian modular $(a*b) \mod n$ adalah dengan melakukan perkalian antara $a$ dan $b$ kemudian menggunakan algoritma pembagian untuk menemukan modulus $n$ dari $a*b$ \citep{applied_crypto}.

    % jelasin montgomery multiplication, basisnya, perkalian, mont reduction
    Selain algoritma sederhana tersebut, algoritma perkalian modular montgomery adalah sebuah algoritma perkalian modular yang lain. Algoritma ini berbasis pada bentuk representasi montgomery, yaitu transformasi nilai $a \mod n$ ke ruang montgomery dengan mengkalikan dengan sebuah integer $R$ sehingga terdapat $a' = aR \mod n$ \citep{mont_mul_bertoni}. Dalam implementasinya, nilai $R$ merupakan nilai $2^k$ sehingga proses perkalian dan pembagian terhadap $R$ dapat dilakukan dengan cepat dengan melakukan \textit{bit shifting}.

    Untuk mendapatkan kembali nilai $a$ dari ruang montgomery, perlu dilakukan reduksi montgomery yang dijelaskan pada Pseudocode \ref{alg:mont_redc}. Reduksi modular montgomery ini lebih efektif dibandingkan dengan proses pembagian biasa. Pembagian dan modulo pada reduksi montgomery dilakukan oleh $R$, karena $R$ merupakan bilangan pangkat dua, maka proses pembagian dapat dilakukan dengan \textit{left shift} dan modulo dapat dilakukan dengan \textit{bit masking}.

    \begin{algorithm}
      \caption{Algoritma Reduksi Montgomery \citep{montmul_original}}
      \label{alg:mont_redc}
      \begin{algorithmic}[1]
        \Statex

        \Function{MontRedc}{$A, R, N$}
            \Let{$Ni$}{$-1/n \mod R$}
            \Let{$k$}{$(A \mod R) * (Ni \mod R)$}
            \Let{$a$}{$(T + kN)/R$}
            \If{$a \geq T$}
                \Let{$a$}{$a - N$}
            \EndIf
            \State \Return{$a$}
        \EndFunction
      \end{algorithmic}
    \end{algorithm}

    Perkalian montgomery merupakan perkalian antara dua bilangan pada ruang montgomery dan juga menghasilkan bilangan pada ruang montgomery. Perkalian dua bilangan pada ruang montgomery secara langsung tidak menghasilkan bilangan pada ruang montgomery, sehingga perlu dilakukan reduksi montgomery pada hasil perkalian tersebut. Persamaan \ref{eq:montmul} menjelaskan pernyataan tersebut lebih lanjut, sementara \textit{Pseudocode} \ref{alg:montmul} menjelaskan algoritma perkalian modular montgomery dalam pseudocode.

    \begin{equation} \label{eq:montmul}
        \begin{split}
            a'b' &= (aR * bR) \mod n\\
                 &= (abR^2) \mod n\\
                 &= (c'R^{-1}) \mod n
        \end{split}
    \end{equation}

    \begin{algorithm}
      \caption{Algoritma Perkalian Montgomery \citep{montmul_original}}
      \label{alg:montmul}
      \begin{algorithmic}[1]
        \Require{$A, B$ merupakan bilangan dalam ruang montgomery.}
        \Statex

        \Function{MulMont}{$A, B, N, R$}
            \Let{$C$}{$A*B$}
            \Let{$C$}{\Call{MontRedc}{$C, R, N$}}
            \State \Return{$C$}
        \EndFunction
      \end{algorithmic}
    \end{algorithm}

    \subsubsection{Perpangkatan Modular}
    Algoritma yang dapat digunakan untuk melakukan perpangkatan modular sama dengan algoritma dalam perpangkatan biasa. Perbedaan dalam perpangkatan modular adalah penggunaan perkalian modular alih-alih perkalian biasa. Proses perkalian modular yang dilakukan telah dijelaskan dalam subbab \ref{sec:modmul}.

    Sebagai contoh, \textit{Pseudocode} \ref{alg:window_modexp} merupakan algoritma perpangkatan modular sliding window yang menggunakan perkalian modular biasa. Untuk menggunakan perkalian montgomery, hanya perlu diubah pemanggilan fungsi ModMul() menjadi fungsi ModMulMont().

    \begin{algorithm}
      \caption{Algoritma Perpangkatan Modular Sliding Window}
      \label{alg:window_modexp}
      \begin{algorithmic}[1]
        \Require{$A = [0..m],$ bilangan bulat dalam representasi array. $E = [0..n-1]$ bilangan bulat dalam basis binary}
        \Statex

        \Function{WindowModExp}{$A, E, M$}
        \Let{$w$}{\Call{Partition}{$E$}}
        \State \Call{PrecomputeA}{$A, w$}
        \Let{$r$}{$1$}
        \For {$i \gets 0$, $i \gets i + 1$ to $length(w)$}
            \For {$j \gets 0$, $i \gets j + 1$ to $length(w[i])$}
                \Let{$r$}{\Call{ModMul}{$r, r, M$}}
            \EndFor
            \Let{$r$}{\Call{ModMul}{$r, a^{w[i]}, M$}}
        \EndFor
        \State \Return{$r$}
        \EndFunction
      \end{algorithmic}
    \end{algorithm}

  \section{Komputasi Paralel}
% \blindtext
% Apa itu
Komputasi paralel adalah sebuah mode komputasi dimana banyak komputasi dijalankan pada waktu yang sama \citep{highly_parallel_computing}. Sebuah proses pada komputasi paralel biasanya dibagi menjadi beberapa subproses kecil yang akan dijalankan secara independen. Sebuah proses kecil tersebut biasa dipanggil \textit{job}, dan pembagian sebuah proses menjadi job merupakan tantangan utama dalam pemrograman paralel.

% Kenapa butuh -> perkembangan processor ke arah sana
Pada awalnya, komputasi paralel hanya digunakan pada superkomputer mengingat keperluan \textit{hardware} yang tinggi yang diperlukan untuk melakukan komputasi paralel. Namun, setelah IBM memperkenalkan IBM POWER4, sebuah processor multicore pertama pada tahun 2001, komputasi paralel dapat dilakukan oleh komputer komersial umum. Intel Platinum D yang diluncurkan pada tahun 2005 merupakan awal dari jajaran multicore processor yang sekarang sudah umum digunakan pada laptop, komputer dekstop, ataupun smartphone yang kita gunakan. \todo{tambahin kutipan}

Perkembangan processor pada dua dekade ke belakang berfokus ke pembuatan jumlah core yang banyak dibandingkan dengan perkembagan kinerja dari single core. Pada tahun 1986 hingga 2002, kinerja single core processor meningkat 50\% per tahun \citep{comp_arch_patterson}. Sementara itu, peningkatan kinerja single core dari tahun 2002 hanya meningkat sebesar 20\% \citep{intro_parallel} dan akan semakin berkurang pada setiap tahunnya. \todo{Tambahin data dari performance increase / intel CPU}

% Kelebihan
Komputasi paralel biasa digunakan untuk memroses banyak data yang tidak berkaitan satu sama lain. Karena itulah komputasi paralel banyak digunakan pada bidang kecerdasan buatan untuk melakukan \textit{training} pada model yang dibutuhkan. Selain pemrosesan data, komputasi paralel unggul dalam melakukan komputasi aritmatika sederhana seperti perhitungan vektor dalam \textit{3D rendering}. Komputasi paralel juga banyak digunakan dalam komputasi pelipatan protein, permodelan iklim, pembuatan obat, serta penelitian energi \citep{intro_parallel}. \todo{tambahin kutipan}

% Apa yang perlu diperhatikan dalam paralel programming
Pada komputasi paralel, terdapat beberapa hal yang perlu diperhatikan yang biasanya tidak menjadi masalah pada komputasi biasa. \textit{Memory management} perlu lebih diperhatikan dibandingkan dengan pemrograman biasa. Penggunaan \textit{memory management} yang tidak baik dapat menyebabkan program tidak berjalan dengan optimal karena program tidak memperhatikan penggunaan cache dengan baik, bahkan program bisa tidak berjalan sama sekali jika terjadi \textit{deadlock}. Selain itu, \textit{debugging} pada program lebih sulit dilakukan karena \textit{programmer} perlu memperhatikan beberapa \textit{routine} yang berjalan secara paralel. Struktur kode yang tidak rapi akan membuat debugging menjadi lebih susah.

% Pitfalls
\citep{structured_parallel_programming} menyatakan bahwa terdapat beberapa hal yang harus dihindari dalam pembuatan program secara paralel. Hal tersebut dapat terjadi karena sinkronisasi antar \textit{job} tidak dilakukan dengan baik. Terlalu banyak sinkronisasi dapat menyebabkan \textit{scaling} program sulit, sementara itu sinkronisasi yang kurang dapat menyebabkan hasil yang dihasilkan program tidak konsisten. Berikut adalah hal yang perlu diperhatikan.
\begin{itemize}
  \item \textit{Race condition}, yaitu kondisi dimana hasil yang dihasilkan dari komputasi paralel tidak konsisten karena kurangnya sinkronisasi antar \textit{job}. Sebagai contoh, job A menulis nilai baru pada memori M, sementara itu job B membaca memori M sebelum job A menulis nilai baru tersebut ketika job B membutuhkan nilai memori M yang terbaru.
  \item \textit{Deadlock} terjadi ketika dua \textit{job} menunggu satu sama lain untuk menggunakan sebuah resource yang sama.
  \item \textit{Scaling} program lebih susah, hal ini terjadi karena sinkronisasi yang terlalu tinggi. Karena sinkronisasi antar job sangat tinggi, program tidak berbeda jauh dengan program yang dijalankan secara sekuensial. Jika program menggunakan lock dan mutex, bisa jadi waktu operasi lock tersebut justru memakan waktu yang lebih lama dibandingkan dengan job yang perlu dijalankan.
  \item \textit{Load imbalance} antar job. Terjadi ketika pembagian pekerjaan dalam job tidak seimbang.  
  \item \textit{Overhead} pembuatan job. Jika jumlah job yang dibuat program terlalu banyak, waktu yang diperlukan dalam pembuatan dan penghancuran job menjadi semakin signifikan. Perlu dicari jumlah job yang cukup sehingga overhead tidak terlalu besar namun program tetap dijalankan secara paralel dengan efektif.
\end{itemize}

  \section{Penelitian Terkait}
% Tentang apa, metode/algoritma, hasil
\subsection{Paralelisasi Big Integer pada GPU}
\citet{gpu_bignum} melakukan penenelitian mengenai penggunaan GPU dalam pemrosesan \textit{multiple precision arithmatic} secara paralel. Dalam penelitiannya, Emmet menganalisis beberapa operasi arimatika, yaitu penjumlahan, pengurangan, serta perkalian untuk bilangan bulat dan bilangan real, perkalian modular, perpangkatan modular, dan pembagian. Emmet berusaha untuk meningkatkan kinerja komputasi big number dalam beberapa faktor, diantaranya jumlah operasi per detik, jumlah operasi per detik per daya yang digunakan, serta jumlah operasi per detik untuk setiap dollar yang digunakan.

% Pendekatan untuk setiap operasi
Emmet melakukan penjumlahan dan pengurangan dengan divide and conquer. Array dalam BIGNUM dibagi menjadi beberapa \textit{chunks}, setiap chunks tersebut kemudian dijumlahkan dengan algoritma penjumlahan biasa, kemudian menggabungkan chunks tersebut dan menyelesaikan hasil \textit{carry over} dari setiap chunks. Operasi perkalian yang dilakukan oleh Emmet berbasis algoritma Strassen pada perkalian dalam format Fast Fourier Transform. Untuk operasi pembagian, Emmet melakukan paralelisasi pada algoritma pembagian pendek.
% TODO: Perlu dijelasin algoritma pembagiannya ga?

% Kok kalo basicnya doang kayak ga impressive

Testing peneltian dilakukan pada dua CPU yaitu Core i5-7400 serta Xeon E5-2690v3, dan tiga GPU yaitu GTX Titan Black, GTX 980, Pascal P100, serta Volta V100. Emmet berhasil membuktikan bahwa untuk tiga faktor yang ditentukan, hasil yang didapat dari GPU lebih baik dibandingkan dengan CPU. Volta V100 memiliki jumlah operasi per detik terbesar, dengan speedup hingga 88 kali lebih besar dari Core i5. Penggunan daya pada Volta 2.2 kali lebih tinggi dibandingkan Xeon, namun Volta masih memiliki jumlah operasi per detik per daya yang lebih besar dibandingkan dengan CPU dan GPU yang lain. Untuk jumlah operasi per detik untuk setiap dollar yang digunakan, intel Core i5 merupakan \textit{socket} yang memiliki nilai terbesar. Namun, Emmet memperkirakan bahwa GTX1080 yang tidak digunakan dalam penelitian ini dapat memberikan nilai operasi per detik untuk setiap dollar yang lebih besar.

\subsection{Fast Modular Squaring}
\citep{drucker_gueron_2019}

% \blindtext
\subsection{Fast Multi-Precision Multiplication}
\citep{hutter_wenger_2018}

% \blindtext

% \blindtext


    \chapter{Analisis}

% Outline:
% 3.1 TLS analysis
% 3.2 BIGNUM analysis


\section{Analisis TLS}
Penggunaan TLS pada sebuah infrastrukur internet dapat membantu penjaminan keamanan dari proses keluar masuknya data. Namun, penggunaan TLS memilki harga yang mahal secara komputasi yang dilakukan, sehingga membuat kinerjanya sendiri lambat. Pengurangan kinerja dari penggunaan TLS sendiri berdampak pada 3.4 hingga 9 kali dibandingkan dengan deployment tanpa penggunaan TLS \citep{perf_tls}. Mengingat beberapa tipe website seperti personal \textit{blog}, portal berita, ataupun \textit{search engine} tidak benar-benar membutuhkan keamanan informasi, tidak sedikit dari situs tersebut yang memilih untuk tidak menggunakan TLS demi mendapatkan kinerja yang maksimal.

\todo[inline]{tambahin persentasi website yang ga pake TLS}

Proses TLS dapat dibagi menjadi dua bagian, yaitu proses TLS \textit{Handshake} yang dilakukan saat membuat sebuah koneksi TLS, serta proses pertukaran data yang dilakukan setelah TLS \textit{Handshake} selesai dilakukan. Berdasarkan eksperimen yang dilakukan, \cite{perf_tls} menyatakan bahwa CPU melakukan lebih banyak pekerjaan untuk menyelesaikan TLS \textit{Handshake} dibandingkan dengan proses pertukaran data. Hal ini menyatakan bahwa TLS Handshake berdampak lebih besar pada kinerja sebuah TLS server dibandingkan dengan pertukaran data.

Penggunaan alogritma kriptografi kunci publik pada proses pertukaran kunci  dalam TLS \textit{Handshake} adalah proses yang membutuhkan komputasi cukup besar, dimana proses tersebut menggunakan 13-59\% dari seluruh proses komputasi pada TLS. Selain itu, proses komputasi kriptografi lainnya seperti RC4, MD5, ataupun pembangkitan nomor random sudah cukup seimbang dan tidak memerlukan banyak komputasi pada TLS (Coarfa et. al, 2006).

\subsection{Implementasi TLS}
% Jelasin salah satu yang popular itu OpenSSL
Terdapat banyak \textit{library} yang mengimplementasikan TLS, diantaranya adalah OpenSSL, BoringSSL, cryptlib, ataupun Java Secure Socket Extension (JSSE). Diantara banyak implementasi tersebut, OpenSSL merupakan implementasi yang paling banyak digunakan. OpenSSL merupakan aplikasi default yang digunakan sebagai implementasi TLS dalam hampir seluruh sistem operasi Linux. Sementara itu, lebih dari 85\% web server yang ada di internet menggunakan sistem operasi Linux \citep{server_os_marketshare}.

% Kenapa pilih OpenSSL sebagai implementasi yang diteliti
Penulis memilih untuk menggunakan OpenSSL sebagai implementasi TLS yang akan dibahas lebih jauh. Selain karena banyak digunakan, OpenSSL memiliki dokumentasi yang lebih baik dibandingkan dengan implementasi yang lain, dengan demikian proses untuk melakukan modifikasi pada \textit{source code} OpenSSL akan lebih mudah. Modul big integer yang dimiliki oleh OpenSSL terpisah dari modul komputasi kriptografi yang lain. Dengan demikian akan memudahkan proses penelitian karena penulis hanya perlu berfokus pada source code di modul itu saja. Selain itu, OpenSSL sudah umum digunakan dalam implementasi HTTPS pada webserver Apache2 ataupun nginx. Hal ini menyebabkan penulis dapat menggunakan tools yang umum digunakan dalam melakukan testing benchmark dalam sebuah aplikasi web seperti Apache Bench.

\section{Analisis Komputasi Big Integer pada OpenSSL}
% \blindtext

\subsection{Algoritma yang Digunakan oleh OpenSSL}
Selain langsung melakukan operasi aritmatika, terdapat juga bagian kode dalam OpenSSL yang melakukan beberapa fungsi lain. Sebagai contohnya, terdapat bagian kode yang melakukan pengecekan apakah OPENSSL\_SMALL\_FOOTPRINT terdefinisi dan melakukan fungsi yang diperlukan untuk mengkonversi tipe data yang digunakan agar cocok untuk versi OpenSSL yang terinstall. Pada subbab ini, bagian kode yang akan dibahas hanyalah bagian kode yang relevan dengan algoritma operasi aritmatika tertentu. Bagian kode yang melakukan tugas lain seperti konfigurasi tipe data ataupun \textit{memory management} tidak akan dibahas.

% Kalo refer ke code, how to cite?
% \blindtext
\subsubsection{Penjumlahan dan Pengurangan}
OpenSSL menggunakan algoritma penjumlahan dan pengurangan standar dalam representasi FRNS seperti yang dibahas pada subbab \ref{sec:add_sub_theory}. OpenSSL melakukan operasi $mod$ dan $div$ dengan manipulasi bit, yaitu melakukan AND terhadap sebuah konstan untuk mendapat nilai $mod$ dan melakukan penggeseran bit untuk mendapatkan nilai $div$. Dengan demikian, OpenSSL dapat memotong overhead yang terjadi dibandingkan dengan melakukan operasi $div$ ataupun $mod$ yang terdapat dalam bahasa C.
% TODO: Cek lagi atas ^^

\subsubsection{Perkalian}
OpenSSL menggunakan algoritma karatsuba++

\subsubsection{Pembagian dan Modulo}
Jelasin pembagian

OpenSSL tidak memiliki fungsi khusus untuk operasi modulo. Karena fungsi untuk operasi pembagian sudah mengembalikan sisa dari pembagian, operasi modulo hanya perlu memanggil fungsi pembagian dengan argumen menggunakan argumen yang telah disesuaikan.

\subsection{Paralelisasi Algoritma}
% \blindtext
\subsubsection{Penjumlahan dan Pengurangan}
\subsubsection{Perkalian}
\subsubsection{Pengurangan}
% etc, ntar ditambahin


    %!TEX root = ../../tugas-akhir.tex
\chapter{IMPLEMENTASI DAN EVALUASI}

  %!TEX root = ../../../tugas-akhir.tex
\section{Implementasi Algoritma Paralel pada Library big integer}
  \subsection{Lingkungan Implementasi} \label{sec:impl_env}
    Sesuai dengan pertimbangan pada subbab \ref{sec:parallel_env}, implementasi algoritma paralel akan dilakukan dengan menggunakan pthread. Implementasi akan dilakukan menggunakan \textit{compiler} GCC versi 5.4.0 dan dijalankan pada sistem operasi Ubuntu 18.04 64-bit. Penggunaan TLS akan diuji pada penggunaan HTTPS pada web server yang diinstall pada sistem operasi. Web server yang digunakan adalah Apache2 dengan menggunakan modul tambahan mod\_ssl. Arsitektur sistem yang digunakan dapat dilihat pada Gambar \ref{fig:openssl_arch}

    \begin{figure}[h]
      \centering
      \includegraphics[width=0.4\textwidth]{resources/img/ch-4/implementation_arch.png}
      \caption{Arsitektur OpenSSL}
      \label{fig:openssl_arch}
    \end{figure}

    % Cek lagi jumlahnya
    Secara default, jumlah maksimal thread maksimum yang akan digunakan oleh OpenSSL untuk menjalankan komputasi big integer secara paralel akan bergantung pada jumlah core yang terdapat pada lingkungan instalasi. Dalam aplikasi yang membutuhkan komputasi yang tinggi setiap thread akan berjalan secara terus menerus. Dengan demikian jumlah thread maksimum akan sama dengan jumlah aplikasi. Namun, aplikasi juga memiliki pilihan konfigurasi untuk menentukan jumlah thread maksimum yang dapat digunakan.
    % \todo{cite disini}

  \subsection{Batasan Implementasi}
    Implementasi dilakukan pada sistem operasi Ubuntu 64 bit. Karena itu, implementasi library big number hanya berfokus pada openssl dengan yang memiliki konfigurasi makro sebagai berikut:

    \begin{enumerate}[label=\roman*.]
      \item BN\_ULONG = unsigned long long
      \item OPENSSL\_SMALL\_FOOTPRINT = false
      \item BN\_MUL\_COMBA = true
      \item BN\_RECURSION = true
      \item BN\_CTX\_POOL\_SIZE = 16
    \end{enumerate}
    \todo{masukin lampiran?}
    Konfigurasi makro tersebut digunakan dalam pemilihan dan manajemen struktur data yang digunakan serta pemilihan algoritma pada bagian tertentu. Sebagai contoh, algoritma yang digunakan dalam perkalian adalah algoritma karatsuba dan algoritma comba pada basis rekursif.


  \subsection{Struktur File \textit{Source Code}}

    \begin{figure}[h]
      \centering
      \includegraphics[width=0.8\textwidth]{resources/img/ch-4/file-tree.png}
      \caption{Struktur Source Code OpenSSL}
      \label{fig:ossl_file_structure}
    \end{figure}

    Struktur \textit{source code} dapat dilihat pada Gambar \ref{fig:ossl_file_structure}. Direktori utama berisi pembagian dari fungsi aplikasi yang ada dalam OpenSSL. Beberapa fungsi tersebut adalah direktori |doc| yang berisi dokumentasi OpenSSL, |crypto| yang berisi library kriptografi, |ssl| yang berisi dari library komunikasi ssl, serta |test| yang berisi unit test yang dimiliki OpenSSL.

    Direktori |crypto| berisi modul-modul yang membentuk libcrypto, dengan setiap modul terbentuk dalam sebuah direktori yang berbeda. Modul |bn| merupakan modul yang mengatasi perhitungan operasi aritmatika big integer. Selain itu, modul |dh| dan |rsa| merupakan modul yang menangani komputasi Diffie-Hellman dan RSA pada OpenSSL.

    Modul |bn| memiliki beberapa submodul masing-masing merupakan file yang berbeda. Setiap submodul sendiri mengatasi fungsi yang terkait dengan submodul tersebut, ataupun operasi yang komputasinya mirip dengan submodul. Sebagai contoh, submodul |bn_add| merupakan submodul yang mengatasi operasi penjumlahan dan pengurangan pada big integer. Selain itu, submodul |bn_mul| merupakan submodul yang mengatasi operasi perkalian serta pemilihan algoritma perkalian yang digunakan dalam OpenSSL.

    Direktori |test| dan |util| merupakan direktori yang digunakan dalam \textit{unit test} OpenSSL. Dalam direktori ini terdapat kasus uji dan script yang digunakan dalam unit test untuk setiap modul untuk melakukan test terhadap fungsionalitas OpenSSL. Penjelasan lebih jauh terhadap pengujian fungsionalitas OpenSSL dapat dilihat pada subbab \ref{sec:func_testing}

  \subsection{Struktur Data Big Integer} \label{sec:bignum_struct}
    Pada OpenSSL, sebuah big integer direpresentasikan dalam struktur data |BIGNUM|. |BIGNUM| terdiri dari sebuah array dengan ukuran dinamis dan beberapa variabel integer yang menyimpan informasi tambahan. Dengan demikian secara teori |BIGNUM| tidak memiliki nilai maksimum. Untuk keperluan paralelisasi, |BIGNUM| dapat digunakan tanpa mengubah strukturnya sedikitpun. BIGNUM sendiri merupakan sebuah \textit{struct} yang memiliki deklarasi sesuai oada Source Code \ref{code:bignum_st}.

    \begin{lstlisting}[caption={Struktur Data bignum}, label={code:bignum_st}]
struct bignum_st {
       BN_ULONG *d;
       int top;
       int dmax;
       int neg;
       int flags;
};
    \end{lstlisting}

    |BN_ULONG| sendiri adalah sebuah makro yang menggantikan |unsigned long| pada komputer 32 bit atau |unsigned long long| pada komputer 64 bit.

    |d| adalah pointer untuk array of integer.

    |top| merupakan index |d| yang terakhir digunakan plus satu.

    |dmax| adalah panjang maksimum array yang telah dibuat. |neg| bernilai satu jika BIGNUM bernilai negatif.

    Setiap kali pembuatan struktur data |BIGNUM| terjadi alokasi memori yang memiliki overhead yang cukup tinggi jika dilakukan berulang ulang \citep{doc_bnctx}. Sementara itu, operasi aritmatika kompleks seperti perkalian dengan algoritma karatsuba, pembagian, atau perpangkatan modulo membutuhkan beberapa struktur |BIGNUM| yang digunakan untuk menyimpan variabel sementara. Struktur data |BN_CTX| menyimpan sejumlah variabel |BIGNUM| yang dapat digunakan dalam operasi aritmatika, dengan demikian program tidak pelu melakukan alokasi memori setiap kali program membutuhkan sebuah struktur |BIGNUM|.

    % jelasin lebih jauh

    \begin{lstlisting}[caption={Struktur bignum\_ctx}, label={code:bignum_ctx}]
struct bignum_ctx {
    BN_POOL pool;
    BN_STACK stack;
    unsigned int used;
    int err_stack;
    int too_many;
    int flags;
};

/* BN_POOL */
typedef struct bignum_pool_item {
    BIGNUM vals[BN_CTX_POOL_SIZE];
    struct bignum_pool_item *prev, *next;
} BN_POOL_ITEM;
typedef struct bignum_pool {
    BN_POOL_ITEM *head, *current, *tail;
    unsigned used, size;
} BN_POOL;

/* BN_STACK */
typedef struct bignum_ctx_stack {
    unsigned int *indexes;
    unsigned int depth, size;
} BN_STACK;
    \end{lstlisting}

    Struktur dari |BN_CTX| dapat dilihat pada Source Code \ref{code:bignum_ctx}. |BN_CTX| terdiri dari kumpulan |BIGNUM| yang disimpan pada |BN_POOL| serta |BN_STACK| yang menyimpan jumlah |BIGNUM| yang digunakan oleh sebuah fungsi. |BN_POOL| merupakan list of array dengan panjang array sebesar |BN_CTX_POOL_SIZE|. |BN_STACK| digunakan untuk menyimpan jumlah BIGNUM yang digunakan dalam sebuah fungsi.

    Fungsi yang menggunakan |BN_CTX| harus memanggil |BN_CTX_start()| sebelum penggunaan |BN_CTX| dan memanggil |BN_CTX_end()| setelah pemanggilan |BN_CTX|. Dua fungsi tersebut akan menyimpan dan menghitung jumlah BIGNUM yang didapat dari pemanggilan |BN_CTX_get()| pada fungsi tersebut.

  \subsection{Modul Operasi Aritmatika}
    \subsubsection{Submodul Penjumlahan dan Pengurangan}
      Modul penjumlahan dan pengurangan terdapat pada file BN\_add.c. Fungsi BN\_add() pada modul ini melakukan pengolahan data pada a dan b seperti mengecek negatif dan mengecek panjang masing-masing array. Hasil pengecekan tersebut akan digunakan untuk melakukan operasi lebih lanjut. Jika a dan b memiliki tanda yang berbeda, akan dipanggil fungsi BN\_usub() selain itu akan dipanggil fungsi BN\_uadd(). BN\_uadd() dan BN\_usub() melakukan pengolahan data pada a dan b sehingga terdapat representasi array yang dapat diolah oleh BN\_add\_words() dan BN\_sub\_words(). Daftar fungsi yang terdapat pada submodul penjumlahan dan pengurangan dapat dilihat pada Tabel \ref{tab:bn_add_func}.

      BN\_add\_words merupakan fungsi yang menerima masukan dua array dengan ukuran yang sama dan menjumlahkannya secara sekuensial. Penerapan algoritma \ref{alg:add} pada OpenSSL terdapat pada fungsi ini.

      \begin{table}[!h]
        \caption{Fungsi dalam submodul bn\_add}
        \label{tab:bn_add_func}
        \begin{tabular}{R{2.8cm}L{10.5cm}}
          \toprule
          \textbf{Header Fungsi} & |int BN_add(BIGNUM *r, const BIGNUM *a, const BIGNUM *b)|    \\ \midrule
          \textit{Deskripsi}     & Menjumlahkan a dan b dan menyimpan hasilnya pada r $(a+b=r)$ \\
          \textit{Prekondisi}    & -                                                            \\
          \textit{Return Value}  & 1 jika fungsi berhasil dilakukan dan 0 jika tidak
          \\ \bottomrule
          \textbf{Header Fungsi} & |int BN_sub(BIGNUM *r, const BIGNUM *a, const BIGNUM *b)|    \\ \midrule
          \textit{Deskripsi}     & Mengurangi b dari a dan menyimpan hasilnya pada r $(a-b=r)$  \\
          \textit{Prekondisi}    & -                                                            \\
          \textit{Return Value}  & 1 jika fungsi berhasil dilakukan dan 0 jika tidak
          \\ \bottomrule
          \textbf{Header Fungsi} & |int BN_uadd(BIGNUM *r, const BIGNUM *a, const BIGNUM *b)|   \\ \midrule
          \textit{Deskripsi}     & Menjumlahkan a dan b dan menyimpan hasilnya pada r $(a+b=r)$ \\
          \textit{Prekondisi}    & $a \geq 0$, $ b \geq 0$                                      \\
          \textit{Return Value}  & 1 jika fungsi berhasil dilakukan dan 0 jika tidak
          \\ \bottomrule
          \textbf{Header Fungsi} & |int BN_usub(BIGNUM *r, const BIGNUM *a, const BIGNUM *b)|   \\ \midrule
          \textit{Deskripsi}     & Mengurangi b dari a dan menyimpan hasilnya pada r $(a-b=r)$  \\
          \textit{Prekondisi}    & $a \geq 0$, $b \geq 0$, $a \geq b$                           \\
          \textit{Return Value}  & 1 jika fungsi berhasil dilakukan dan 0 jika tidak
          \\ \bottomrule
        \end{tabular}

      \end{table}

    \subsubsection{Modul Perkalian}
      \begin{table}[h]
        \caption{Fungsi dalam submodul bn\_add}
        \begin{tabular}{R{2.8cm}L{10.5cm}}
          \toprule
          \textbf{Header Fungsi} & |int BN_mul(BIGNUM *r, const BIGNUM *a, const BIGNUM *b, BN_CTX *ctx)|                                                                                                \\ \midrule
          \textit{Deskripsi}     & Mengalikan $b$ pada $a$ dan menyimpan hasilya dalam $r, (r = a * b)$.                                                                                                 \\
          \textit{Prekondisi}    & -                                                                                                                                                                     \\
          \textit{Return Value}  & 1 jika fungsi berhasil dilakukan dan 0 jika tidak
          \\ \bottomrule
          \textbf{Header Fungsi} & |void bn_mul_normal(BN_ULONG *r, BN_ULONG *a, int na, BN_ULONG *b, int nb)|                                                                                           \\ \midrule
          \textit{Deskripsi}     & Perkalian $a$ dan $b$ dengan menggunakan algoritma perkalian panjang, dengan $na$ adalah panjang $a$ dan $nb$ adalah panjang $b$.                                     \\
          \textit{Prekondisi}    & -                                                                                                                                                                     \\
          \textit{Return Value}  & 1 jika fungsi berhasil dilakukan dan 0 jika tidak
          \\ \bottomrule
          \textbf{Header Fungsi} & |void bn_mul_recursive(BN_ULONG *r, BN_ULONG *a, BN_ULONG *b, int n2 int dna, int dnb, BN_ULONG *t)|                                                                  \\ \midrule
          \textit{Deskripsi}     & Perkalian $a$ dan $b$ dengan menggunakan algoritma perkalian karatsuba. $n2$ adalah panjang hasil perkalian, dengan $dna = length(a) - n2$ dan $dnb = length(b) - n2$ \\
          \textit{Prekondisi}    & $length(r) = 2*n2$. $ length(t) = 2*n2$. $n2 = 2^k, k \in \mathbb{Z}. $                                                                                               \\
          \textit{Return Value}  & 1 jika fungsi berhasil dilakukan dan 0 jika tidak
          \\ \bottomrule
        \end{tabular}

      \end{table}

    \subsubsection{Submodul Pembagian}
      \begin{table}[h]
        \caption{Fungsi dalam submodul bn\_add}
        \begin{tabular}{R{2.8cm}L{10.5cm}}
          \toprule
          \textbf{Header Fungsi} & |BN_div(BIGNUM *dv, BIGNUM *rm, const BIGNUM *num, const BIGNUM *divisor, BN_CTX *ctx)|                                                                                                       \\ \midrule
          \textit{Deskripsi}     & Membagi num dengan divisor, hasil pembagian disimpan sebagai dv dan sisa pembagian disimpan sebagai rm. Baik div maupun rm bisa menjadi NULL jika hasil atau sisa pembagian tidak dibutuhkan. \\
          \textit{Prekondisi}    & -                                                                                                                                                                                             \\
          \textit{Return Value}  & 1 jika fungsi berhasil dilakukan dan 0 jika tidak
          \\ \bottomrule
          \textbf{Header Fungsi} & |int bn_left_align(BIGNUM *num)|                                                                                                                                                              \\ \midrule
          \textit{Deskripsi}     & Normalisasi BIGNUM $num$ agar $num > \beta/2$                                                                                                                                                 \\
          \textit{Prekondisi}    & -                                                                                                                                                                                             \\
          \textit{Return Value}  & 1 jika fungsi berhasil dilakukan dan 0 jika tidak
          \\ \bottomrule
        \end{tabular}

      \end{table}

    \subsubsection{Submodul Asm}
      \begin{table}[h]
        \caption{Fungsi dalam submodul bn\_add}
        \begin{tabular}{R{2.8cm}L{10.5cm}}
          \toprule
          \textbf{Header Fungsi} & |BN_ULONG bn_add_words(BN_ULONG *a, const BN_ULONG *a, const BN_ULONG *b, int n)|  \\ \midrule
          \textit{Deskripsi}     &                                                                                    \\
          \textit{Prekondisi}    & -                                                                                  \\
          \textit{Return Value}  &
          \\ \bottomrule
          \textbf{Header Fungsi} & |BN_ULONG bn_sub_words(BN_ULONG *r, const BN_ULONG *a, const BN_ULONG *b, int n)|  \\ \midrule
          \textit{Deskripsi}     &                                                                                    \\
          \textit{Prekondisi}    & -                                                                                  \\
          \textit{Return Value}  &
          \\ \bottomrule
          \textbf{Header Fungsi} & |BN_ULONG bn_mul_words(BN_ULONG *rp, const BN_ULONG *ap, int num, BN_ULONG w)|     \\ \midrule
          \textit{Deskripsi}     &                                                                                    \\
          \textit{Prekondisi}    & -                                                                                  \\
          \textit{Return Value}  &
          \\ \bottomrule
          \textbf{Header Fungsi} & |BN_ULONG bn_mul_add_words(BN_ULONG *rp, const BN_ULONG *ap, int num, BN_ULONG w)| \\ \midrule
          \textit{Deskripsi}     &                                                                                    \\
          \textit{Prekondisi}    & -                                                                                  \\
          \textit{Return Value}  &
          \\ \bottomrule
          \textbf{Header Fungsi} & |BN_ULONG bn_div_words(BN_ULONG h, BN_ULONG l, BN_ULONG d)|                        \\ \midrule
          \textit{Deskripsi}     &                                                                                    \\
          \textit{Prekondisi}    & -                                                                                  \\
          \textit{Return Value}  &
          \\ \bottomrule
        \end{tabular}

      \end{table}

  %!TEX root = ../../../tugas-akhir.tex
\section{Evaluasi}
\subsection{Tujuan Pengujian}
Pada penelitian ini pengujian dilakukan untuk melakukan evaluasi apakah modifikasi terhadap perangkat OpenSSL merupakan modifikasi yang baik. Pengujian dilakukan terhadap dua bagian dari OpenSSL, yaitu pengujian terhadap fungsionalitas OpenSSL serta pengujian terhadap kinerja OpenSSL.

Pengujian fungsional dilakukan untuk memastikan bahwa fungsionalitas dari OpenSSL tetap berjalan dengan baik setelah dilakukan modifikasi terhadap \textit{source code}. Pengujian fungsional dilakukan terhadap operasi artimatika big integer yang telah diubah untuk memastikan hasil dari operasi aritmatika tersebut masih merupakan hasil yang benar.

Pengujian kinerja dilakukan untuk menjawab rumusan masalah poin pertama dan ketiga pada subbab \ref{sec:rumusan_masalah}. Pengujian kinerja akan menjawab apakah kinerja TLS dapat meningkat setelah menggunakan algoritma paralel dalam operasi aritmatika big integer. Selain itu, pengujian kinerja akan menunjukkan apakah algoritma paralel yang digunakan dalam operasi aritmatika merupakan algoritma yang tepat atau tidak.

\subsection{Pengujian Fungsional}

\subsubsection{Lingkungan Pengujian}
Lingkungan pengujian yang digunakan dalam pengujian fungsional sama dengan lingkungan implementasi yang telah dijelaskan dalam \ref{sec:impl_env} namun tanpa penggunaan Apache sebagai web server. Pengujian fungsional hanya membutuhkan \textit{source code} OpenSSL serta testcase yang digunakan.

\subsubsection{Skenario Pengujian}
% OpenSSL udah punya tc
OpenSSL telah memiliki kakas yang dapat dilakukan untuk melakukan unit test. Kakas yang digunakan merupakan gabungan dari \textit{perl script} dan kode dalam bahasa C. \textit{Perl script} akan menjalankan kode dalam bahasa C dan memberikan kasus uji yang sesuai untuk pengujian yang dilakukan. Kode dalam bahasa C akan melakukan \textit{parsing} terhadap kasus uji yang ada, menjalakan fungsi tertentu, kemudian mencocokkan hasil dari fungsi tersebut terhadap kasus uji yang diberika.

% jelasin gimana cara run -> makefile
Pengujian fungsional dilakukan melalui makefile dengan command |make test|. Pengujian dilakukan setelah build aplikasi selesai dilakukan, namun sebelum dilakukannya instalasi aplikasi pada sistem. Command |make test| melakukan pengujian terhadap seluruh fungsionalitas OpenSSL, mulai dari library big integer, komputasi kriptografi, sistem sertifikat, serta pembuatan koneksi TLS.

% testcase yang dipilih kayak gimana, kenapa
Kasus uji untuk setiap operasi aritmatika disimpan pada |openssl/test/recipes/10-test_bn_data|. Kasus uji untuk setiap operasi aritmatika memiliki file terpisah dengan operasi artimatika yang lain. Sebagai contoh, kasus uji untuk operasi perkalian disimpan pada file |bnmul.txt| pada direktori pengujian. Contoh kasus uji yang digunakan untuk pengujian fungsional dapat dilihat pada Lampiran \ref{sec:functional_testcase}.

\subsubsection{Hasil Pengujian}
Pengujian fungsionalitas pada OpenSSL berhasil dilakukan. Tidak ada fungsionalitas dari OpenSSL yang rusak setelah dilakukan modifikasi pada \textit{source code} OpenSSL. \textit{Output} dari pengujian fungsional yang dilakukan dapat dilihat pada Lampiran XXX. \todo{tambahin}

\subsection{Pengujian Kinerja}

\subsubsection{Lingkungan Pengujian}

Kakas yang digunakan untuk testing adalah ApacheBench. Sesuai dengan namanya, ApacheBench dapat melakukan benchmarking pada web server. ApacheBench (ab) akan mengirim sejumlah request dan menghasilkan data kinerja web server yang dipilih. Data yang dihasilkan pleh ab diantaranya adalah jumlah byte yang dikirim antar web server dan ab, jumlah request per detik yang dapat ditangani server, waktu rata-rata yang digunakan untuk menangani sebuah request, serta waktu maksimum dan minimum yang digunakan dalam sebuah request.

Pengujian dilakukan pada \textit{cloud provider} DigitalOcean. Cloud dipilih karena proses pembuatan server on-demand dengan spesifikasi tertentu dapat dilakukan dengan mudah dan dalam waktu yang singkat. Penggunaan server dengan jumlah core yang tinggi yang dibutuhkan dapat dilakukan dengan menggunakan DigitalOcean CPU-Optimized Droplet. Droplet tersebut mendapatkan dedicated CPU sehingga komputasi paralel pada OpenSSL dapat berjalan dengan menggunakan CPU pada kapasitas maksimum. Arsitektur yang digunakan pada pengujian dapat dilihat pada Gambar \ref{fig:testing_arch}.

\begin{figure}[h]
  \centering
  \includegraphics[width=0.8\textwidth]{resources/ch-4/testing_arch.png}
  \caption{Arsitektur Lingkungan Pengujian}
  \label{fig:testing_arch}
\end{figure}

Pengujian menggunakan empat Droplet, yaitu lab, test01, test16, dan test32. Setiap droplet terhubung melalui DigitalOcean Private Networking, dengan demikian latency antar droplet dapat dikurangi hingga seminimum mungkin. Droplet test01, test16, dan test32 merupakan droplet tempat diinstallnya Apache dan OpenSSL. Arsitektur aplikasi yang terinstall pada test01, test16, dan test32 sama dengan arsitektur implementasi pada Gambar \ref{fig:openssl_arch}. Droplet lab digunakan sebagai tempat melakukan testing dan tempat dijalankannya kakas ApacheBench. Komunikasi antar droplet lab dan personal workstation dilakukan dengan ssh. Tabel \ref{tab:droplet_specs} menjelaskan spesifikasi yang dimiliki setiap droplet.

\begin{table}[ht]
\caption{Spesifikasi Droplet} % title of Table
\label{tab:droplet_specs}
\centering % used for centering table
\begin{tabular}{|| c | c c c c ||} % centered columns (4 columns)
\hline\hline %inserts double horizontal lines
Spesifikasi\#1 & lab & test01 & test16 & test32\\[0.5ex] % inserts table
%heading
\hline % inserts single horizontal line
Memori    & 1GB   & 1GB   & 32GB  & 64 \\
vCPU      & 1     & 1     & 16    & 32 \\
SSD Disk  & 25GB  & 25GB  & 200GB & 400 \\ [1ex] % [1ex] adds vertical space
\hline\hline %inserts single line
\end{tabular}
\label{table:nonlin} % is used to refer this table in the text
\end{table}

\subsubsection{Skenario Pengujian}
Pengujian dilakukan dengan menjalankan ApacheBench pada node lab dengan tujuan web yang dicek test01, test16, dan test32. ApacheBench menghasilkan beberapa data, namun kita hanya peduli pada data berikut.
% \todo{cite website apachebench}
\begin{enumerate}[label=\roman*.]
  \item \textit{Time taken for tests}, yaitu waktu dari koneksi pertama dibuat sampai response terakhir diterima.
  \item \textit{Requests per second}, yaitu jumlah request yang ditangani dalam satu detik
  \item \textit{Time per request}, yaitu waktu rata-rata untuk sebuah request.
  \item \textit{Min \& max connection times}, yaitu waktu maksimum dan minimum yang digunakan oleh sebuah request.
\end{enumerate}

Pengujian dengan ApacheBench dilakukan terhadap beberapa versi perangkat OpenSSL. Versi pertama adalah OpenSSL vanilla yang belum dimodifikasi, versi kedua adalah OpenSSL dengan paralelisasi modul penjumlahan dan pengurangan, versi ketiga adalah OpenSSL dengan paralelisasi modul perkalian, versi keempat adalah OpenSSL dengan paralelisasi modul pembagian, dan versi kelima adalah OpenSSL dengan semua paralelisasi semua modul operasi aritmatika.

\subsubsection{Hasil dan Analisis Pengujian}

\paragraph{Operasi Penjumlahan}
\begin{figure}
  \centering
  \begin{tikzpicture}
  	\begin{axis}[
  		xlabel=Panjang Data $(bit)$,
  		ylabel={Waktu $(ms)$}
  	]
    \addplot table [x=size, y=c1, col sep=comma] {resources/csv/example.csv};
    \addplot table [x=size, y=c16, col sep=comma] {resources/csv/example.csv};
    \addplot table [x=size, y=c32, col sep=comma] {resources/csv/example.csv};
  	\end{axis}
  \end{tikzpicture}
  \caption{Data Penjumlahan}
  \label{fig:add_data}
\end{figure}
%
% \begin{tikzpicture}
% \begin{loglogaxis}[
% title=Convergence Plot,
% xlabel={Degrees of freedom},
% ylabel={$L_2$ Error},
% grid=major,
% legend entries={$d=2$,$d=3$,$d=4$},
% ]
% \addplot table {data_d2.dat};
% \addplot table {data_d3.dat};
% \addplot table {data_d4.dat};
% \end{loglogaxis}
% \end{tikzpicture}

\paragraph{Operasi Pengurangan}
\paragraph{Operasi Perkalian}
\paragraph{Operasi }

  % Langsung data speedup aja, data raw simpen di lampiran
  %
  % \begin{table}[ht]
  % \caption{Test table} % title of Table
  % \centering % used for centering table
  % \begin{tabular}{c c c c} % centered columns (4 columns)
  % \hline\hline %inserts double horizontal lines
  % Case & Method\#1 & Method\#2 & Method\#3 \\ [0.5ex] % inserts table
  % %heading
  % \hline % inserts single horizontal line
  % 1 & 50 & 837 & 970 \\ % inserting body of the table
  % 2 & 47 & 877 & 230 \\
  % 3 & 31 & 25 & 415 \\
  % 4 & 35 & 144 & 2356 \\
  % 5 & 45 & 300 & 556 \\ [1ex] % [1ex] adds vertical space
  % \hline %inserts single line
  % \end{tabular}
  % \label{table:nonlin} % is used to refer this table in the text
  % \end{table}


    %!TEX root = ../../tugas-akhir.tex
\chapter{Penutup}

\section{Kesimpulan}
% \blindtext

\section{Saran}
% \blindtext


    % -------------------------------------------------------------------------%
    % Daftar Pustaka
    % -------------------------------------------------------------------------%
    \renewcommand{\bibname}{Daftar Pustaka}
    \cleardoublepage
    \phantomsection
    \addcontentsline{toc}{chapter}{DAFTAR PUSTAKA}
    \bibliography{references}


    %-------------------------------------------------------------------------%
    % Lampiran
    %-------------------------------------------------------------------------%
    \chapterfont{\Large}
    \titleformat{\chapter}[block]
      {\Large\bfseries}
      {\chaptertitlename\ \thechapter}{0pt}
        {\Large\bfseries}
    \titlespacing*{\chapter}{0pt}{-25pt}{20pt}

    % \setcounter{chapter}{0}

    \addtocontents{toc}{\protect\setcounter{tocdepth}{-2}}

    \setcounter{secnumdepth}{-2}
    % \begin{appendices}
    % \renewcommand{\chaptername}{Lampiran}
    % \renewcommand{\thechapter}{\Alph{chapter}. }

    \chapter{Daftar Fungsi dalam Modul bn yang Berkaitan dengan Operasi Aritmatika}\label{sec:bn_func_all}
Berikut adalah fungsi yang merupakan operasi aritmatika dalam modul big integer.
\VerbatimInput[label=\fbox{list-func.txt}]{resources/bn_func_all.txt}

    \appendix{Pemanggilan Library Big Number} \label{sec:bn_func_call}
% \todo{Benerin nomor lampiran}
Berikut adalah hasil menjalankan \textit{grep} dengan pattern fungsi-fungsi Big Number pada modul Diffie Hellman dan RSA pada OpenSSL.

\paragraph{Diffie Hellman}
Modul Diffie Hellman pada OpenSSL terdapat pada direktori |openssl/crypto/dh|

\VerbatimInput[label=\fbox{dh.txt}]{resources/bn_grep/grep_result_dh.txt}

\paragraph{RSA}
Modul RSA pada OpenSSL terdapat pada direktori |openssl/crypto/rsa|

\VerbatimInput[label=\fbox{rsa.txt}]{resources/bn_grep/grep_result_rsa.txt}

    %!TEX root = ../../tugas-akhir.tex
\clearpage
\paragraph{Kasus Uji Fungsional} \label{sec:functional_testcase}

Berikut adalah contoh kasus uji yang digunakan dalam pengujian fungsional modul bn dalam OpenSSL. Test uji yang sebenarnya digunakan tidak dimasukkan pada bagian ini karena jumlahnya terlalu banyak.

\subparagraph{Penjumlahan dan Pengurangan}
\VerbatimInput[label=\fbox{bn\_sum.txt}]{resources/text/bn_testcase/bn_sum.txt}

\subparagraph{Perkalian}
\VerbatimInput[label=\fbox{bn\_mul.txt}]{resources/text/bn_testcase/bn_mul.txt}

\subparagraph{Pembagian}
\VerbatimInput[label=\fbox{bn\_div.txt}]{resources/text/bn_testcase/bn_div.txt}

\subparagraph{Perpangkatan}
\VerbatimInput[label=\fbox{bn\_exp.txt}]{resources/text/bn_testcase/bn_exp.txt}
%
\subparagraph{Perkalian Modular}
\VerbatimInput[label=\fbox{bn\_modmul.txt}]{resources/text/bn_testcase/bn_modmul.txt}
%
\subparagraph{Perpangkatan Modular}
\VerbatimInput[label=\fbox{bn\_modexp.txt}]{resources/text/bn_testcase/bn_modexp.txt}


    % \end{appendices}

\end{document}
